%%%---PREAMBLE---%%%%%%%%%%%%%%%%%%%%%%%%%%%%
\documentclass[twoside,12pt,final]{ucthesis-CA2012}

% fix for pandoc 1.14
\providecommand{\tightlist}{%
  \setlength{\itemsep}{0pt}\setlength{\parskip}{0pt}}

% From {rticles}

%--- Packages ---------------------------------------------------------
\usepackage[lofdepth,lotdepth,caption=false]{subfig}
\usepackage{fancyhdr}
\usepackage{amsmath, amssymb, graphicx}
\usepackage{xspace}
\usepackage{braket}
\usepackage{color}
\usepackage{setspace}
\usepackage{fancyvrb}
\usepackage{array}
\usepackage{ifxetex,ifluatex}
\usepackage{etoolbox}

%% for the per mil symbol
\usepackage[nointegrals]{wasysym}

% more attractive tables
\usepackage{booktabs}
\usepackage{xcolor}
\usepackage{tabu}
\usepackage{tabularx}
\usepackage{lscape}
\usepackage{longtable}
\usepackage{titlesec}


\usepackage[nostamp]{draftwatermark}
% % Use the following to make modification
\SetWatermarkText{DRAFT}
\SetWatermarkLightness{0.95}

%---New Definitions and Commands------------------------------------------------------

\newtheorem{theorem}{Jibberish}

\bibliography{references}

\hyphenation{mar-gin-al-ia}

% from uw_template.tex

% commands and environments needed by pandoc snippets
% extracted from the output of `pandoc -s`
%% Make R markdown code chunks work

\ifxetex
  \usepackage{fontspec,xltxtra,xunicode}
  \defaultfontfeatures{Mapping=tex-text,Scale=MatchLowercase}
\else
  \ifluatex
    \usepackage{fontspec}
    \defaultfontfeatures{Mapping=tex-text,Scale=MatchLowercase}
  \else
    \usepackage[utf8]{inputenc}
  \fi
\fi
\DefineShortVerb[commandchars=\\\{\}]{\|}
\DefineVerbatimEnvironment{Highlighting}{Verbatim}{commandchars=\\\{\}}
% Add ',fontsize=\small' for more characters per line
\newenvironment{Shaded}{}{}
\newcommand{\KeywordTok}[1]{\textcolor[rgb]{0.00,0.44,0.13}{\textbf{{#1}}}}
\newcommand{\DataTypeTok}[1]{\textcolor[rgb]{0.56,0.13,0.00}{{#1}}}
\newcommand{\DecValTok}[1]{\textcolor[rgb]{0.25,0.63,0.44}{{#1}}}
\newcommand{\BaseNTok}[1]{\textcolor[rgb]{0.25,0.63,0.44}{{#1}}}
\newcommand{\FloatTok}[1]{\textcolor[rgb]{0.25,0.63,0.44}{{#1}}}
\newcommand{\CharTok}[1]{\textcolor[rgb]{0.25,0.44,0.63}{{#1}}}
% Added lines below after bug and recommendation : https://tex.stackexchange.com/questions/660787/undefined-control-sequence-for-normaltok-or-specialchartok

\newcommand{\ConstantTok}[1]{\textcolor[rgb]{0.53,0.00,0.00}{{#1}}}
\newcommand{\SpecialCharTok}[1]{\textcolor[rgb]{0.25,0.44,0.63}{{#1}}}
\newcommand{\AttributeTok}[1]{\textcolor[rgb]{0.49,0.56,0.16}{{#1}}}
\newcommand{\StringTok}[1]{\textcolor[rgb]{0.25,0.44,0.63}{{#1}}}
\newcommand{\CommentTok}[1]{\textcolor[rgb]{0.38,0.63,0.69}{\textit{{#1}}}}
\newcommand{\OtherTok}[1]{\textcolor[rgb]{0.00,0.44,0.13}{{#1}}}
\newcommand{\AlertTok}[1]{\textcolor[rgb]{1.00,0.00,0.00}{\textbf{{#1}}}}
\newcommand{\FunctionTok}[1]{\textcolor[rgb]{0.02,0.16,0.49}{{#1}}}
\newcommand{\RegionMarkerTok}[1]{{#1}}
\newcommand{\ErrorTok}[1]{\textcolor[rgb]{1.00,0.00,0.00}{\textbf{{#1}}}}
\newcommand{\NormalTok}[1]{{#1}}
\newcommand{\OperatorTok}[1]{\textcolor[rgb]{0.00,0.44,0.13}{\textbf{{#1}}}}
\newcommand{\BuiltInTok}[1]{\textcolor[rgb]{0.00,0.44,0.13}{\textbf{{#1}}}}
\newcommand{\ControlFlowTok}[1]{\textcolor[rgb]{0.00,0.44,0.13}{\textbf{{#1}}}}

\ifxetex
  \usepackage[setpagesize=false, % page size defined by xetex
              unicode=false, % unicode breaks when used with xetex
              xetex,
              colorlinks=true,
              linkcolor=blue]{hyperref}
\else
  \usepackage[unicode=true,
              colorlinks=true,
              linkcolor=blue]{hyperref}
\fi
\hypersetup{breaklinks=true, pdfborder={0 0 0}}
\setlength{\parindent}{0pt}
\setlength{\parskip}{6pt plus 2pt minus 1pt}
\setlength{\emergencystretch}{3em}  % prevent overfull lines
\setcounter{secnumdepth}{0}

%---Set Margins ------------------------------------------------------
\setlength\oddsidemargin{0.25 in} \setlength\evensidemargin{0.25 in} \setlength\textwidth{6.25 in} \setlength\textheight{8.50 in}
\setlength\footskip{0.25 in} \setlength\topmargin{0 in} \setlength\headheight{0.25 in} \setlength\headsep{0.25 in}

%%%---DOCUMENT---%%%%%%%%%%%%%%%%%%%%%%%%%%%%
\begin{document}

%=== Preliminary Pages ============================================
\begin{ucfrontmatter}

  %%%%%%%%%%%%%%%%%%%%%%%%%%%
  % TITLE PAGE INFORMATION %  modified to meet UCDavis, R. Peek, 2018
  %%%%%%%%%%%%%%%%%%%%%%%%%%%

  \title{Organic Seed Systems}
  \author{Liza Wood}

\report{DISSERTATION} 
  \degree{DOCTOR OF PHILOSOPHY} 
  \degreemonth{September} \degreeyear{2023}
  \chair{Mark Lubell}  % this is your advisor
  \othermemberA{Tyler Scott} % This is a member of your committee
  \othermemberB{Neil McRoberts} % This is a member of your committee
  \othermemberC{} % This is a member of your committee
  \numberofmembers{3} % should match the number of entries above (chair + othermembers)
  \field{ECOLOGY}
  \campus{DAVIS}
  
	\maketitle
	
	% APPROVAL AND COPYRIGHT
	% \approvalpage % AS OF 2018 Fall, don't need this additional page if use cover page for signatures
	\copyrightpage

  %%%%%%%%%%%%%%%%%%%%%%%%%%%
  % DEDICATION PAGE INFORMATION %
  %%%%%%%%%%%%%%%%%%%%%%%%%%%

 %  \begin{dedication}

 %    \vspace*{20ex}

 %    \begin{center}
 %    \begin{large}

 %      

 %    \end{large}

 %    \end{center}

 %\end{dedication}
  
  % ACKNOWLEDGEMENTS
\begin{acknowledgements}
    ``My acknowledgments''
  \end{acknowledgements}
  %%%%%%
  % CV % Not required, add if you need
  %%%%%%
%   \begin{vitae}
%     \addcontentsline{toc}{chapter}{Curriculum Vitae}
% 
%     \begin{vitaesection}{Education}
%     \vspace{-0.1cm}
%     \item [2018]	Ph.D. in Environmental Science and Management (Expected), University of California, Santa Barbara.
%     \item [2010]	MESM in in Environmental Science and Management, University of California, Santa Barbara.
%     \item [2007]	B.S. in Ecosystem Science and Policy and Biology, University of Miami
%     \end{vitaesection}
% 
%     \textbf{Publications}
% 
%     Anderson, S.C., Cooper, A.B., Jensen, O.P., Minto, C., Thorson, J.T., Walsh, J.C., Afflerbach, J., Dickey‐Collas, M., Kleisner, K.M., Longo, C., Osio, G.C., Ovando, D., Mosqueira, I., Rosenberg, A.A., Selig, E.R., n.d. Improving estimates of population status and trend with superensemble models. Fish and Fisheries 18, 732–741. https://doi.org/10.1111/faf.12200
% 
%  Burgess, M.G., McDermott, G.R., Owashi, B., Reeves, L.E.P., Clavelle, T., Ovando, D., Wallace, B.P., Lewison, R.L., Gaines, S.D., Costello, C., 2018. Protecting marine mammals, turtles, and birds by rebuilding global fisheries. Science 359, 1255–1258. https://doi.org/10.1126/science.aao4248
% 
% Costello, C., Ovando, D., Clavelle, T., Strauss, C.K., Hilborn, R., Melnychuk, M.C., Branch, T.A., Gaines, S.D., Szuwalski, C.S., Cabral, R.B., Rader, D.N., Leland, A., 2016. Global fishery prospects under contrasting management regimes. PNAS 113, 5125–5129. https://doi.org/10.1073/pnas.1520420113
% 
% \end{vitae}

	%%%%%%%%%%%%%%%%%%%%%%%%%%%
  % ABSTRACT %
  %%%%%%%%%%%%%%%%%%%%%%%%%%%
  \begin{abstract}
    \addcontentsline{toc}{chapter}{Abstract}

    ``Overall abstract?''

    %\abstractsignature
  \end{abstract}
  % TABLE OF CONTENTS
	\tableofcontents

%	%  \listoftables
%  %
%  %  \listoffigures
%  
\end{ucfrontmatter}
\begin{ucmainmatter}

\hypertarget{introduction}{%
\chapter*{Introduction}\label{introduction}}
\addcontentsline{toc}{chapter}{Introduction}

Short tying it all together

\hypertarget{the-double-edged-sword-of-flexible-environmental-policy-the-case-of-the-seed-loophole-in-certified-organic-production}{%
\chapter{The double-edged sword of flexible environmental policy: The case of the seed loophole in certified organic production}\label{the-double-edged-sword-of-flexible-environmental-policy-the-case-of-the-seed-loophole-in-certified-organic-production}}

\chaptermark{organic seed policy}

\hypertarget{introduction-1}{%
\section{1. Introduction}\label{introduction-1}}

(Why environmental policy is important, environmental challenges, etc.?)

Effective environmental policy has to walk the line between being
stringent enough to improve environmental outcomes, while also flexible
enough so that the policies don\textquotesingle t backfire (Barreiro-Hurle et al.~2023;
Sunstein 2017). Overly stringent policies may cause industries to
opt-out/exit or find methods for creative compliance if the burden is
too high (Bartel and Barclay 2011). Yet, too flexible of policies can
also be taken advantage of by free-riders, enabling industries to shirk
environmental responsibilities while still gaining (Prakash and Potoski
2007). In this paper we ask: \textbf{Do flexible environmental standards spare
firms from undue burden or enable free-riders?}

To answer this question we study a flexible voluntary environmental
policy: the organic certification in the United States and its so-called
seed loophole. Organic standards require growers to comply with a wide
range of practices and sourcing rules that support soil health and
biodiversity and reduce environmental impact. Using certified organic
seed is one of the sourcing requirements, however, the policy permits
exceptions if the variety is not ``commercially available''. This
exception, mirrored in organic policies around the world, gives
flexibility to growers with their seed sourcing to ideally limit the
burden when organic seeds are unavailable.

This paper analyzes organic growers\textquotesingle{} reliance on the organic seed
exception using three cross-sectional surveys over the last fifteen
years. We measure organic growers\textquotesingle{} use of conventional seed (i.e.~their
reliance on the loophole) as well as their experiences with seed
sourcing barriers, perceptions about organic seed, and their farm
operational attributes. Using these data, we model the relationship
between conventional seed use and grower attributes to test hypotheses
about the sparing and/or enabling effect of the flexible policy.

We find evidence that the flexibility of the organic seed loophole is
both supporting growers who need it, as well as enabling more
\textquotesingle conventionalized\textquotesingle/\textquotesingle practical\textquotesingle{} organic farmers. The use of
conventional seed is associated with growers who experience the highest
barriers to sourcing organic seed, meaning that the policy does help
reduce the burden on growers who cannot find suitable seed in organic
form. At the same time, the policy also allows a certain profile of
farmers to avoid organic seed use. Farmers of this profile appear to be
those that represent the \textquotesingle conventionalization\textquotesingle{} of organic agriculture
-- larger, less diverse farm operations who don\textquotesingle t see organic seed as
important for the success or integrity of organic production.

These results track with the idea that actors have different motivations
for compliance with environmental policy (Bartel and Barlcay 2011) and
that policy design needs to account for different motivations (Piniero
2020). Though policy design research tends to treat firms as purely
rational actors (Timbe and Winter 2015, Prakash and Potoski), farmers
especially have different motivations and embrace business models that
balance economic and environmental values (Thompson, Brown et al.~2021).
These come through as two identities of organic, practical and
principled (Darnhofer), which affect the ways that growers engage with
the policy. Not only can reckoning with these identities help strike a
better balance within the policy, it also opens up a broader
conversation about the different kinds of \textquotesingle sustainability\textquotesingle{} that the
policy wants to support. (Draft: For organic seed, this means creating a
mix of policies that support traditional expansion of organic seed
availability and enforcement of the policy, as well as alternative
initiatives for principled, grassroots programs)

\hypertarget{background}{%
\section{2. Background}\label{background}}

\hypertarget{environmental-policy-flexibility}{%
\subsection{2.1 Environmental policy flexibility}\label{environmental-policy-flexibility}}

Environmental policy has to strike a balance between stringency and
flexibility. Stringent environmental policies often take a command and
control approach, where firms are required to meet specific criteria
such as defined contributions, reduction targets, or technology
standards (Barreiro-Hurle et al.~2023; Pettersson and Soderholm 2014).
These types of regulatory measures reduce uncertainty in the achievement
of environmental outcomes so long as enforcement is in place (Pineiro et
al.~2020). However, stringency has its drawbacks. The increase of
regulation may reduce competitiveness of firms (), counteract voluntary
behavior (Barreiro-Hurle et al.~2023), prompt relocation to an area with
more lenient policies (), or resistance/creative compliance (Bartel and
Barclay 2011).

Flexible environmental policies, on the other hand, use a range of tools
to motivate firms via incentives to improve environmental outcomes
(Gunningham 1999; Pineiro et al.~2020), and at times permit
exceptions/derogations if the regulation is overly burdensome
(Romero-Castro et al.~2022). The benefits of flexible policies include
reduced tension between firms and government (), as well as room for
innovation to ultimately improve environmental outcomes (Loonie et al.
2011; Yuan and Zhang 2020). The drawback of flexible policy tools,
however, is that participation can be low, adherence needs to be
monitored to avoid free-riding (Prakash and Poroski 2007), and there is
less certainty around the environmental benefits (Stuart, Benveniste,
and Harris 2014).
Furthermore, vague criteria for exceptions/derogations like the use of
\textquotesingle best available technology\textquotesingle{} gives wide latitude to firms\textquotesingle{} discretion
(Romero-Castro et al.~2022).

In this paper we are interested in environmental policy that provides
flexibility in the form of exceptions. For example, the European Union
industrial emissions directive mandates firms to reduce emissions using
\textquotesingle best available technology\textquotesingle, but grant derogations if the costs of
complying (as calculated by the firm) are too high (Romero-Castro et al.
2022, Soderholm et al.~2021). In the United States, regulation of manure
discharge by animal feeding operations has several exceptions to
regulation based on nutrient management plans (Rosov et al.~2020). Who
takes advantage of these exceptions is an open question, prompting our
research question: Do flexible environmental standards spare firms from
undue burden or enable free-riders?

\hypertarget{organic-agricultural-standards}{%
\subsection{2.2 Organic agricultural standards}\label{organic-agricultural-standards}}

\hypertarget{flexibility-of-organic-certification}{%
\subsubsection{2.2.1 Flexibility of organic certification}\label{flexibility-of-organic-certification}}

Organic certification is a voluntary program that agricultural producers
and food processors opt into, and are then monitored according to the
certification standards. These kinds of programs are considered
high-cost clubs -- stringent voluntary standards that gain participants
premiums (Prakash and Potoski 2007). However, the certification
standards are generally accompanied by listed exceptions for using
non-organic products. These exceptions are proposed to reduce excessive
burdens and barriers facing growers and processors (e.g.~NOSB 2004, NOSB
2006) in an otherwise stringent and often challenging certification
process (Flaten et al.~2010, Carter er al.~2022).

One such exception in the organic label is related to seed sourcing.
Organic standards generally require organic growers to use organic seed.
However, so-called \textquotesingle loopholes\textquotesingle{} are often put into place (Endres,
Liveseed, Padel), where organic growers are allowed to use conventional
seed if organic seed is not commercially available. This kind of
exception is shared in organic standards around the world, for example
the United States\textquotesingle{} National Organic Program\footnote{System resources (also called \textquotesingle elements\textquotesingle) and IS functions
  generally map onto one another (Binz et al.~2016; Musiolik et al.
  2011). Musiolik et al.~2011 describe: ``System elements provide
  positive externalities such as public financial support, the
  deliberate diffusion of knowledge or the creation of legitimacy.
  These contributions at the system level can be allocated to the
  system functions and indicate how important the identified system
  elements for TIS development are.\textquotesingle{} (p.~1919 2011); And while some
  elements such as \textquotesingle value chain coordination\textquotesingle{} are not in the
  traditional list of functions (Bergek et al.~2008), in this paper we
  generally think of''functions'' and ``system resources'' as
  interchangeable, but try to primarily use the \textquotesingle resources\textquotesingle{} language
  over the \textquotesingle functional\textquotesingle{} language.}, United Kingdom\textquotesingle s Soil
Association\footnote{We believe that the structures observed by Musiolik et al.~(2020)
  are limited based on the authors\textquotesingle{} \textquotesingle ego-centric\textquotesingle{} interview methods
  (Chung et al 2005) which relied largely on the perspective of key
  actors to describe how networks are formed by system builders only.
  A comprehensive sampling method that gathers data on all actors\textquotesingle{}
  connections would be able to more fully represent the network
  structures and likely show \textquotesingle closure\textquotesingle{} of triangles in the cases of
  partner mode structures.}, European Union\textquotesingle s Organic Standard, and the East Africa
Organic Standard\footnote{We do note, however, that some policy network theory proposes the
  opposite relationship between network structure and the need for
  network-level competencies like trust and addressing complex
  problems. Specifically, Berardo and Scholz (2010) propose that when
  the challenges facing the network have low complexity and therefore
  require fewer resources and low-risk, then centralized,
  open-triangle structures prevail. As complexity increases and there
  is more of a need for creative solutions, transitive,
  closed-triangle structures prevail. This contrasting view represents
  the ongoing debate in network studies on the social processes
  representing by the classic open and closed triangle motifs.} all have equivalent exceptions based availability.

In the United States, ``the goal {[}of the exception{]} is to promote the
continued growth and improvement in organic seed production and
subsequent usage by organic growers, without hurting or putting undo
{[}sic{]} burdens on growers.'' (NOSB 2008). The intention is that this
exception can excuse farmers who experience challenges, particularly as
the organic seed market develops (Hubbard 2011, 2016). While there is
evidence that the seed system is growing through increased public
research spending and the development of new organic varieties (Hubbard
et al.~2021; Liveseed?), there is still widespread uncertainty as to how
the seed loophole is used by organic growers (Hubbard, Zystro, and Wood
2022).

\hypertarget{different-models-of-organic-farming}{%
\subsubsection{2.2.2 Different models of organic farming}\label{different-models-of-organic-farming}}

To develop an understanding of how farmers are using the organic seed
loophole, we first recognize that farmers (and firms more generally)
have different motivations (Bartel and Barclay 2011). Many policy design
studies tend to treat firms as purely rational actors (Timbe and Winter
2015). As Prakash and Potoski describe it, \textquotesingle firms choose to join a
voluntary club (and produce the environmental externalities it requires)
in response to the costs of externality production and the returns from
affiliating with the voluntary club brand\textquotesingle{} (p.~287, 2007). But in
reality, firm-level decisions are made by actors with varying risk
tolerances, entrepreneurial initiatives, and environmental values and
strategies (CITES, Ambec). This is especially true for agricultural
settings, where securing economic returns exists on balance with other
values and operational attributes (Prokopy et al.~2018, Floress et al.
2017).

Farmers make decisions based on a gradient of values ranging from
\textquotesingle farm-as-business\textquotesingle{} to \textquotesingle environmental stewardship\textquotesingle, though the two
are not mutually exclusive (Thompson et al.~2015, dolaglu?).
Additionally, operational attributes such as farm size can influence the
willingness and ability to innovate. In some cases large operations may
be the ones that have the resources to take on risk and experiment with
new practices ((Feder and Umali
1993), \href{https://www.zotero.org/google-docs/?broken=rs6XVC}{Dunn et al.,
2016}), but in other
cases it may hinder the willingness to try something new at a large
scale, especially if there is no visible economic return (Napier 2000,
(Buttel, Jr, and Larson
1990)).

Variation in grower motivations is especially prominent in organic
agriculture. Organic farming originated as a grassroots movement,
centered around holistic environmental and social farming principles
(Youngberg \& DeMuth,
2013)(\url{https://www.zotero.org/google-docs/?broken=hg5nRL}). This vision
of sustainability emphasizes soil health, biodiversity, social justice,
and community food systems as a contrast to the industrialized model of
conventional farms (ibid, Coleman). But when codified into practice as
certified organic standards, there was quickly a trend towards
\textquotesingle conventionalization\textquotesingle{} of organic. In this conventionalization, growers
participate in organic practices (as defined by the standard) but not
organic principles (as defined by the movement\textquotesingle s broader
social-environmental ethos), mainly to gain the price premium (Darnhofer
et al.~2010, Guthman).

The division between \textquotesingle principled\textquotesingle{} and \textquotesingle practical\textquotesingle{} organic farmers
has created tension around the identity of organic farming (von Shlen
2007, Coleman 2001). This tension has come to bear throughout the
process of setting and revising the certification standards, such as the
bi-annual discussion of the National Organic Standards Board, where
exceptions are reviewed and updated (DuPuis). Indeed, these discussions
reflect a wider conversation about the vision of what organic is and the
appropriate path to achieve it. Though the division between
\textquotesingle principled\textquotesingle{} and \textquotesingle practical\textquotesingle{} organic farmers does not fully
represent the diversity of farmer profiles involved in organic (Guthman,
Darnhofer), they help us typify different ways that growers may be
engaging with the organic standard and its flexibility.

\hypertarget{hypotheses}{%
\section{3. Hypotheses}\label{hypotheses}}

We use the case of the organic agricultural standard to ask: Do flexible
environmental standards spare farmers from undue burden or enable
free-riders? We propose two sets of hypotheses.

The first proposition tests whether environmental policies add
flexibility in order to reduce excessive burden on firms. Organic
farming is certainly challenging (Sahm et al.~2012), and barriers to
complying with the standard are high, especially in the early phases of
the certification\textquotesingle s establishment. As such, we expect to see evidence
that growers who are relying on the seed loophole (i.e.~planting
conventional seed) are those who experience the greatest barriers to
accessing organic seed. Furthermore, we expect that as the organic seed
market develops over time, fewer growers are using conventional seed.
\begin{quote}
H1a. Conventional seed use will be highest among growers experiencing
more significant organic seed sourcing barriers

H1b. Conventional seed use will decline over time
\end{quote}
The second proposition tests whether policy flexibility enables some
farmers to take advantage of the policy\textquotesingle s leniency (Prakash; Stuart
2014). In the organic seed policy, though guidance for growers and
certifiers has tried to define and operationalize the \textquotesingle commercial
availability\textquotesingle{} exception (NOSB 2018), there is still opportunity for
shirking. We propose that those particularly inclined to take advantage
of the policy\textquotesingle s leniency are farmers fitting the \textquotesingle practical\textquotesingle{} profile,
given that they have little principled motivation to adopt organic seed.
In this case, we expect to see evidence that growers relying on the seed
loophole are those who align with that profile -- farmers who place low
value on organic seeds\textquotesingle{} role in organic integrity and have operations
that resemble conventional farms: large acreage and low crop diversity.
\begin{quote}
H2a. Conventional seed use will be highest among those with who place
low value on organic seeds\textquotesingle{} contribution to organic integrity

H2b. Conventional seed use will be highest on large farms

H2c. Conventional seed use will be highest on low-diversity farms
\end{quote}
\hypertarget{methods}{%
\section{4. Methods}\label{methods}}

\hypertarget{case-usda-national-organic-program-and-organic-seed}{%
\subsection{4.1 Case: USDA National Organic Program and organic seed}\label{case-usda-national-organic-program-and-organic-seed}}

In this paper we focus on the case of the US Department of Agriculture
(USDA) certified organic label. The organic movement in the US began
prominently organizing in the 1970s, and it is through their collective
action that the Organic Foods Production Act was passed as part of the
1990 Farm Bill and ultimately ratified in 2002 (Youngberg). In the
arrangement, the USDA National Organic Program (NOP) determines the
standards, though they are advised by the National Organic Standards
Board (NOSB) -- a panel of industry representatives that meets regularly
to continuously discuss exceptions and general guidance with input from
the public.

The general idea of certified organic crop production is that growers
use only biological inputs (e.g.~no pesticides or genetically modified
seeds) to create an agro-ecosystem with healthy soil, diverse crop
rotations, and minimal environmental impact (USDA NOP 2015). From start
to finish of the certified organic supply chain, inputs should be
organic to \textquotesingle protect the integrity\textquotesingle{} of the label, including seeds. In
practical terms, organic seeds are those that are cultivated under
certified organic conditions, meaning that using organic seed
contributes to environmental outcomes by having more acreage managed
under organic practices. The principle of organic seed, however, extends
even further. Organic seeds ought to support genetic and crop diversity
maintenance, be adapted to organic growing conditions, and suitable for
regional environments (Hubbard et al., 2022; Rohe et al.,
2022)(\url{https://www.zotero.org/google-docs/?broken=pvMEt9}). Furthermore,
organic breeding is often collaborative, where seed sovereignty and
social needs are included as project considerations (Colley, 2022;
Dawson et al.,
2011)(\url{https://www.zotero.org/google-docs/?broken=6ndb0v}). Altogether,
organic seed is a foundational part of organic production, especially
for more principled organic growers (Hubbard et al.~2022).

There are exceptions to several of the organic requirements, as
mentioned in Section 2.2.1. Of focus in this paper is that USDA NOP
organic growers are allowed to use conventional seed if ``an equivalent
organically produced variety is not commercially available'' (7 CFR §
205.204). Third-party organic certifiers, or accredited certifying
agencies, are responsible for evaluating organic growers\textquotesingle{} compliance
with the organic standard, and whether or not a farm qualifies for the
organic seed exemption is at their discretion. Now that more than twenty
years have passed since the ratification of the NOP, we draw on data to
understand trends in seed use and the attributes of growers that have
been unable to source organic seed. Such an assessment can help
policymakers rethink the organic seed policy as the US organic standard
enters into its third decade.

\hypertarget{data-organic-grower-survey}{%
\subsection{4.2 Data: Organic grower survey}\label{data-organic-grower-survey}}

This research combines three cross-sectional surveys of organic seed
producers in the United States. These data were collected in 2009-2011,
2014-2016, and 2019-2021, which we will refer to as the 2010, 2015, and
2020 time periods. The nation-wide organic grower survey, orchestrated
by the national non-profit organization Organic Seed Alliance, covers
four topics: Farm profile, use of organic seed, barriers to organic seed
and sourcing, and attitudes towards organic seed.

In each data collection period, grower surveys were distributed using
one or both of the following methods: a random sample from the USDA NOP
INTEGRITY database and a convenience sample from an open-access web
survey. Responses from both the combined random and convenience samples
resulted in a total of 899 responses in 2010, 1,162 responses in 2015,
and 760 responses in 2020 that had identifiable regions and crops. The
points on the maps in Figure 1.1 represent the distribution of our
survey respondents. We cannot associate a response rate with our
convenience sampling, but the response rates for the random samples were
25\% in 2015 and 22.5\% in 2020 (convenience sampling only in 2010). Of
these responses, 545 (2010), 799 (2015) and 408 (2020) were complete and
therefore usable for data analysis for a total of 1,770.

\{width=``3.9791666666666665in''
height=``3.4657261592300963in''\}
\begin{figure}

{\centering \includegraphics[width=0.8\linewidth]{../organicseed_adoption/figures/figure1} 

}

\caption{Map of the United States depicting survey responses (points) over three time periods, 2010 (purple), 2015 (blue), 2020 (green).}\label{fig:unnamed-chunk-9}
\end{figure}
\hypertarget{quantitative-analysis}{%
\subsection{4.3 Quantitative analysis}\label{quantitative-analysis}}

\hypertarget{variables}{%
\subsubsection{4.3.1 Variables}\label{variables}}

Descriptions of the variables used in our model are described in Table
1.1. The key dependent variable in our model is \emph{acreage planted to
conventional seed}. The survey asks respondents: ``Last year, what
approximate percentage of total acreage of {[}vegetable/forage/field
crops{]} was planted with certified organic seed?''. The response to this
question is a continuous variable between 0 and 100, which was converted
into a proportion. Then we subtracted this value from one to represent
the proportion of acreage planted to conventional seed.

Our analysis includes eight independent variables for predicting
adoption. The first five variables are relevant to our first hypotheses.
This includes four variables describing \emph{barriers} to sourcing organic
seed. Growers were asked, ``Over the last three years, how much were
each of the following a factor in your decision NOT to purchase organic
seed?''. The top reasons include: Variety is not available as organic,
lack of desirable genetic traits, insufficient quantity of seed, and
processor/buyer requires or supplies varieties that are not available in
organic form. Responses were Likert-scale ratings from ``Not a factor''
(1) to ``Significant factor'' (4). Additionally, the \emph{survey year}
captures how the organic seed market has changed over time.

The next three variables are relevant to our second hypothesis. Organic
growers\textquotesingle{} principled \emph{value} \emph{of organic seed} are derived based on
respondents\textquotesingle{} agreement with the statement: ``Organic seed is important
in maintaining the integrity of organic food production.'' Agreement
ranged from strongly disagree (1) to strongly agree (5). \emph{Farm size},
measured as acreage ranging from 1-17,000 acres, is a representation of
scale and farm resources. And \emph{crop diversity} is the number of crop
categories growers produce, which include field, vegetable, and forage
crops.

Last, we include four variables in our model as controls: variables we
know to influence adoption but are not of theoretical interest to this
paper. These include the amount of \emph{seed sourced via seed saving or
trading}, represented as the percentage of seeds that farmers report
sourcing from either their own farm or from other farmers, ranging
between 0 and 100 percent. \emph{Certifiers request organic seed use} is the
growers\textquotesingle{} reported experience with certifiers\textquotesingle{} regulatory enforcement.
Growers were asked the yes-no question: ``Over the last three years has
your certifier requested that you take greater steps to source organic
seed?''. Last, we include the \emph{region} in which the farmer is located
and the \emph{main crop} of that grower based on their percent acreage
planted to that crop, to account for variation in the development and
availability of organic seed in these different subsystems.

TABLE1.1 need to insert

\hypertarget{modeling-approach}{%
\subsubsection{4.3.2 Modeling approach}\label{modeling-approach}}

We use a beta regression model to estimate the relationships between
organic growers\textquotesingle{} use of conventional seed and their attributes. The
beta distribution is best suited for our model given that the dependent
variable of conventional seed use is bounded by 0 and 1, representing
the proportion of acreage. Prior to fitting the model, we transformed
the data in two ways. First, the acreage proportion data was truncated
by 0.001 so that observations fall within (but not equal to) the 0-1
boundaries of a beta distribution (INLA Documentation: Beta). Second,
all independent variables are mean-centered and scaled for easier
interpretation of coefficients. This way, the values of the coefficients
represent the proportion increase in conventional seed used based on a
unit change from the mean of the variable.

We ran our analysis using the INLA package (Rue et al.~2009) available
in R Statistical software (R Core Team 2023). INLA uses an Integrated
Nested Laplace Approximation -- a deterministic Bayesian method for
estimating its models. Analysis is available at
\href{https://github.com/liza-wood/organicseed_adoption}{{[}github.com/liza-wood/organicseed\_adoption{]}}.
Variable selection is informed largely by theory rather than model
fitting exercises. However, we compare several models and provide
measures of model fit in Appendix A-1.

\hypertarget{descriptive-analysis}{%
\subsection{4.4 Descriptive analysis??}\label{descriptive-analysis}}

\hypertarget{results}{%
\section{5. Results}\label{results}}

\hypertarget{state-of-organic-seed}{%
\subsection{5.1 State of organic seed}\label{state-of-organic-seed}}

Across the three time periods of our survey (2010, 2015, 2020), the mean
values and their standard deviations are reported in Table 1.2.
Conventional seed use has been moderately low since the inception of the
organic certification. The mean proportion of organic crop acreage
planted to conventional seed decreased from 43\% to 32\% between 2010 and
2015, then to 29\% in 2020, with high standard deviations. These numbers
suggest that most farmers are planting at least some organic seed,
though those planting 100\% organic seed represented only 21\%, 27\%, and
26\% of growers in 2010, 2015, and 2020, respectively.

The most significant barrier that growers report as limiting their
organic seed sourcing is that a specific variety is not available in
organic form, which is ranked greater than 3 (between moderate and
significant factor) in all three survey periods. The barriers of
insufficient quantity and desirable genetic traits score an average of
2.6 (between low and moderate factor) in 2010, and that score decreases
by 2020. Buyer requirements are the least significant barrier to
sourcing organic seed, but it does remain a consistent, low ranked
factor over time.

Across the three survey periods, growers\textquotesingle{} perceived value of organic
seed as important to the integrity of organic has remained consistently
high. In the 2010 survey, the average agreement score with the statement
that organic seed is important to the success of organic production was
4 (somewhat important), which increased to 4.3 in 2015 and 2020. The
operational profile of organic growers matches the profile of
agricultural production more generally in that farm sizes are increasing
((Sumner 2014)). The
average farm size over the three survey periods was 180, 275, and 346
acres. This increase is most pronounced in field and forage crops.
Further, the number of crop categories planted by growers has been
declining, with an average of 1.7 crop types in 2010 down to an average
of 1.3 in 2020.

The relationship between certifiers and organic growers has shifted over
the years, as it appears that organic certifiers are becoming less
strict. When non-compliant growers (i.e.~growers not already using 100\%
organic seed) were asked whether or not their certifier had requested
the increase their organic seed use, 61\% responded yes in 2010,
decreasing to 56\% in 2015 and 46\% in 2020. This decline suggests that
certifiers are becoming less insistent to ensure that growers are taking
measures to source organic seed. Further, reliance on non-commercial
seed sources, via seed saving and seed exchange, has decreased from 23\%
and 24\% in 2010 and 2015 to 15\% in 2020.
\begin{table}

\caption{\label{tab:unnamed-chunk-11}Mean values and standard deviations (reported in pathenses) for the key variables from the organic grower survey for each time period}
\centering
\resizebox{\linewidth}{!}{
\begin{tabular}[t]{llll}
\toprule
Questions & stats2011 & stats2016 & stats2021\\
\midrule
Proportion of acreage planted to conventional seed & 0.43 (0.38) & 0.32 (0.36) & 0.29 (0.36)\\
Barrier: Organic availability & 3.55 (0.95) & 3.05 (1.22) & 3.19 (1.15)\\
Barrier: Insufficient quantity of organic seed & 2.62 (1.29) & 2.22 (1.23) & 2 (1.16)\\
Barrier: Lack of desirable genetic traits & 2.63 (1.33) & 2.15 (1.26) & 2.31 (1.27)\\
Barrier: Buyer requirements & 1.87 (1.26) & 1.63 (1.13) & 1.9 (1.25)\\
\addlinespace
Values organic seed & 4.03 (1.07) & 4.29 (0.96) & 4.32 (0.89)\\
Farm size (acres) & 181.05 (342.15) & 275.16 (645.05) & 346.98 (999.54)\\
Vegetable Crop Acreage & 27.58 (90.34) & 62.79 (427.81) & 12.3 (47.81)\\
Field Crop Acreage & 178.21 (347.15) & 259.32 (490.56) & 351.51 (812.27)\\
Forage Crop Acreage & 98.96 (124.74) & 122.03 (197.63) & 172.11 (548.85)\\
\addlinespace
Crop diversity & 1.66 (0.64) & 1.5 (0.6) & 1.25 (0.56)\\
Certifiers request organic seed use & 0.61 () & 0.56 () & 0.46 ()\\
Seed saved or traded (\%) & 22.54 (29.95) & 23.67 (31.69) & 14.63 (24.05)\\
Number of respondents & 545 (545) & 799 (799) & 408 (408)\\
\bottomrule
\end{tabular}}
\end{table}
\hypertarget{section}{%
\subsection{}\label{section}}

\hypertarget{policy-flexibility-for-those-facing-barriers}{%
\subsection{5.2 Policy flexibility for those facing barriers}\label{policy-flexibility-for-those-facing-barriers}}

In this section we present model results related to our first set of
hypotheses, which propose that environmental policies add flexibility in
order to reduce excessive burden on organic producers. We present the
mean coefficient estimates and credible intervals of our model results
in Figure 1.2. Full model results are available in Appendix A-2.

Of the four organic seed sourcing barriers facing organic growers, three
have a significant positive relationship with planting conventional
seed. The barriers related to organic seed variety availability, lacking
desirable genetic traits, and limitations in choice via buyer
requirements have effect sizes of 0.14, 0.13, and 0.11, respectively. In
other words, for each unit higher that a grower ranks the barrier of
organic variety availability, for example, the proportion of acreage
planted to convention seed increases by a mean estimate of 0.14 (14\%).
This estimate is based on all other variables being held constant at
their mean or baseline, so for example, these effects are calculated
based on having an average farm size, number of crops, and value rating
of organic seed, among other variables. The barrier related to
insufficient quantity of an organic variety was not significantly
related to conventional seed use.

Regarding survey year, we find that conventional seed use significantly
decreased over time. In 2015, the coefficient estimate is -0.17,
suggesting that compared to the baseline data in 2010, growers are
planting 17\% less of their acreage to conventional seed. Likewise in
2015, growers are planting 33\% less of their acreage to conventional
seed.

\{width=``6.5in'' height=``3.25in''\}
\begin{figure}

{\centering \includegraphics[width=1\linewidth]{../organicseed_adoption/figures/figure2} 

}

\caption{Coefficient estimates and credible intervals for the beta regression assessing the proportion of total acreage of a growers main crop planted with conventional seed.}\label{fig:unnamed-chunk-12}
\end{figure}
These results support our first hypotheses: H1a) Conventional seed use
will be highest among growers experiencing more significant organic seed
sourcing barriers and H1b) Conventional seed use will decline over time.
Together, they suggest that the flexibility of the organic seed policy
is being used by those with the highest seed sourcing burden, and as the
organic seed market develops over time, growers rely less on the seed
policy exception.

\hypertarget{practical-farmer-profiles-rely-more-on-policy-flexibility}{%
\subsection{5.3 \textquotesingle Practical\textquotesingle{} farmer profiles rely more on policy flexibility}\label{practical-farmer-profiles-rely-more-on-policy-flexibility}}

Next we address our second set of hypotheses, which propose that policy
flexibility enables some farmers to take advantage of the policy\textquotesingle s
leniency. A grower\textquotesingle s rating of organic seed\textquotesingle s value has a significant
negative relationship with planting conventional seed. Our model
estimates an effect size of -0.2, meaning that for every unit higher
that a grower ranks the value of organic seed, the acreage planted to
conventional seed decreases by 20\%. All else held equal -- so for
example farmers who experience the same barriers and have the same farm
sizes -- the principled belief that organic seed is important to the
integrity of organic farming has a considerable effect on how much of
acreage is planted to conventional versus organic seed.

Regarding operational characteristics, farm size has a significant
positive relationship with planting conventional seed. For every unit
larger a farm is, growers plant 11\% more of their acreage to
conventional seed. Crop diversity has a negative relationship with
planting conventional seed, with an effect size of -0.07, but credible
intervals that span between -0.14 and 0.01.

These results support all three of our second hypotheses: H2a)
Conventional seed use will be highest among those with who place low
value on organic seeds\textquotesingle{} contribution to organic integrity, H2b)
Conventional seed use will be highest on large farms, and H2c)
Conventional seed use will be highest on low-diversity farms. These
results suggest that certain farmers\textquotesingle{} profiles -- notably those that
align with a \textquotesingle practical\textquotesingle{} organic grower -- rely more on the organic
seed policy exception than more \textquotesingle principled\textquotesingle{} organic growers.

The crop and regional controls in our model highlight considerable
variation in adoption based on these different organic subsystems.
Forage crop growers plant an estimated 85\% more of their acreage to
conventional seed compared to field crop growers, while vegetables crop
growers plant an estimated 21\% more. Regionally, compared to the
baseline of the North Central region, growers from both the West and the
Northeast regions plant significantly more of their acreage to
conventional seed, with coefficient estimates of 0.35 and 0.24,
respectively. The estimate for the Southern region is also higher than
the baseline (0.12), though the credible intervals of this estimate
cross over zero.

We take results related to our second set of hypotheses and visualize
the differences between two typified farmer profiles: practical and
principled. A practical farmer profile is defined by a low value of
organic seed (25th percentile), large farm size (75th percentile), and
low crop diversity (25th percentile). A principled farmer profile is the
opposite, characterized by a high value of organic seed (75th
percentile), small farms (25th percentile), and high crop diversity
(75th percentile). We calculate these profiles for each region in the
United States and use the example of vegetable crops, with all other
variables fixed at their median value. Using these two profiles we
estimate their predicted acreage planted to conventional seed using our
model.

Figure 1.3 visualizes model predictions of conventional seed use by
hypothetical vegetable growers from these two farmer profiles across
regions in the United States. From the predicted values we see a
consistent difference between the two farm profiles: practical farmers
rely more on the seed policy loophole, planting between 20-39\% more
acreage to conventional seed than principled farmers in 2010, 22-73\% in
2015, and 8-20\% in 2020. The standard errors around these estimates are
plotted, but are so small that they cannot be seen in the figure. We run
these same predictions for Field and Forage crops (available in Appendix
A-3) and observe even higher differences between the two farm profiles
over the three time periods.

\{width=``6.5in'' height=``3.7083333333333335in''\}
\begin{figure}

{\centering \includegraphics[width=1\linewidth]{../organicseed_adoption/figures/figure3} 

}

\caption{Predicted mean estimates and their credible intervals (not visible due to small size) for vegetable crop growers of two farmer profiles. Estimates are made for each region of the US, represented by the different colored points (see legend). The points represent the mean estimates and lines connecting points highlight the difference between practical and principled growers in each time period.}\label{fig:unnamed-chunk-13}
\end{figure}
\hypertarget{descriptive}{%
\subsection{5.4 Descriptive??}\label{descriptive}}

\hypertarget{discussion-and-conclusion}{%
\section{6. Discussion and conclusion}\label{discussion-and-conclusion}}

\hypertarget{flexibility-in-the-organic-standard-is-double-edged}{%
\subsection{6.1 Flexibility in the organic standard is double-edged}\label{flexibility-in-the-organic-standard-is-double-edged}}

We find support for both our hypotheses related to the role of
environmental policy flexibility in the organic standard: the seed
loophole is being used by those with the highest seed sourcing burden,
as well as those with more \textquotesingle practical\textquotesingle{} farm profiles. These results
suggest that the flexibility of the organic policy is both sparing firms
from undue burden and enabling free-riders. Simply put, the standard\textquotesingle s
flexibility is a double-edged sword, as pointed out by plant breeder
John Navazio in an interview about the seed loophole (Roseboro, 2018).

What do I actually contribute? (struggling here)
\begin{itemize}
\item
  These results move beyond a singular assessment: flexible
  \textgreater{} market-based policies are good because they support innovation
  \textgreater{} (Ramanathan) or flexible voluntary standards are bad (Angus 2015).
  \textgreater{} It generates a more nuanced discussion around how to tweak policy
  \textgreater{} to account for the different motivations of growers.
\item
  These results track with the idea that actors have different
  \textgreater{} motivations for compliance with environmental policy (Bartel and
  \textgreater{} Barlcay 2011) and that policy design needs to account for
  \textgreater{} different motivations (Piniero 2020). Though policy design
  \textgreater{} research tends to treat firms as purely rational actors (Timbe and
  \textgreater{} Winter 2015, Prakash and Potoski), farmers especially have
  \textgreater{} different motivations and embrace business models that balance
  \textgreater{} economic and environmental values (Thompson, Brown et al.~2021).
  \textgreater{} We make the point that firms, farmers, and people generally
  \textgreater{} respond to environmental policy stringency (and flexibility)
  \textgreater{} differently, and so tools need to be designed to account for this
  \textgreater{} (Pinero et al.~2020)..
\end{itemize}
This is also an innovation system/innovation policy problem
\begin{itemize}
\item
  By identifying that there are barriers to sourcing, this calls upon
  \textgreater{} innovation policy for organic seed. Namely, to get the balance
  \textgreater{} right, the organic seed standard may require complementary
  \textgreater{} policies that support innovation to build a stronger organic seed
  \textgreater{} innovation system more generally (Rogge and Rechiart). In short,
  \textgreater{} we need a wider mix of policies if we really want to improve
  \textgreater{} organic seed sourcing.
  \begin{itemize}
  \tightlist
  \item
    Basically we need to look beyond single policies and think more
    \textgreater{} broadly about the mix that they\textquotesingle re in
  \end{itemize}
\item
  Though the standard supports organic seed market development and
  \textgreater{} innovation by requiring its use, it simultaneously creates
  \textgreater{} challenges for incentivizing entrepreneurial activity due to this
  \textgreater{} exception, and leaves the decision to certifiers for assessing
  \textgreater{} commercial availability and viable alternatives (Renaud et al.,
  \textgreater{} 2016)(\url{https://www.zotero.org/google-docs/?broken=bElDy3}).
\end{itemize}
\hypertarget{reckoning-with-identities-of-environmental-movements}{%
\subsection{6.2 Reckoning with identities of environmental movements}\label{reckoning-with-identities-of-environmental-movements}}

While finding the right policy tool is central to improving
environmental outcomes (Gunn), it bypasses broader discussion about the
identities that motivate firms, farmers, and people generally to
participate in \textquotesingle sustainability transitions\textquotesingle{} (). Our results highlight
a tension in the organic movement, between practical and principled
farmers, that exists across efforts for social and environmental change
more generally (Smith 2012).

On one hand, participants in grassroots, alternative sustainability
movements often do not want to replicate the systems they developed as
an alternative to (Smith and Raven; Pasucci). For example, farms like
Bee Heaven Farm in Florida produce mixed vegetables for direct sales on
five acres, and grow out seed for use on their operation. Though
organic, farms like these are re-asserting their ethic through
initiatives like the ``Real Organic'' label, that oppose what is
perceived as the industrial co-opting of the organic standard
(\href{http://www.realorganicproject.org}{{[}www.realorganicproject.org{]}}).
This is true for some parts of the organic movement (Youngberg) but also
movements like the open source software initiative (Jain et al.~2023),
open-source seed initiatives (Montenegro de Wit), community energy
initiatives (Smith 2013), and Hackerlabs (Smith).

On the other hand, participants in more traditional sustainability
initiatives want to scale to replace the mainstream modes of production
in order to have the widest impact (Smith and Raven). For example, the
Cal-Organic farm has grown from a quarter of an acre of organic
production to one of the nation\textquotesingle s largest organic vegetable providers,
in part due to merger with conventional operations and industrial supply
chain connection with Whole Foods
(\href{https://calorganicfarms.com/our-story/}{{[}https://calorganicfarms.com/our-story/{]}}).
Likewise, Bayer CropScience, one of the largest agri-chemical and seed
companies in the world, has recently announced its entry into the
organic seed market (Bayer 2021). This same model is aspired to for
several alternative energy technologies like wind () and fuel cells ().

By and large, environmental sustainability initiatives are designed and
evaluated based on the metrics of the latter, traditional approach. As a
result, the focus is largely on getting incentive structures right so
that pro-environmental behavior and innovation can flourish and scale
(Prakash and Potoski). Yet, our results point to the coexistence of both
alternative and traditional visions of sustainability transitions
through the clear difference in seed sourcing behavior by principled and
practical farmer profiles. Based on these findings, we propose that
designing a well-balanced environmental policy needs to reckon with
these multiple, sometimes competing identities. This reckoning includes
recognition of how different policies cater to different identities and
visions within sustainability transitions.

\hypertarget{organic-policy-recommendations}{%
\subsection{6.3 Organic policy recommendations}\label{organic-policy-recommendations}}

In this section we propose recommendations related to the United States
organic standard, recognizing the multiple identities within organic
farming. Recommendations relate to the organic seed policy itself, as
well as innovation policy more broadly that shapes the way organic seed
is developed and made available.

Make more seed available
\begin{itemize}
\item
  Support innovation via: Continue funding for research and organic
  \textgreater{} variety development and support seed producer education and
  \textgreater{} expansion
  \begin{itemize}
  \tightlist
  \item
    Not just through traditional pipelines like plant patenting, but
    \textgreater{} also grassroots initiatives like commons-based
    \textgreater{} breeding/open-source seed initiative
  \end{itemize}
  \texttt{\{=html\}\ \textless{}!-\/-\ -\/-\textgreater{}}
  \begin{itemize}
  \item
    Direct resources towards a diverse portfolio of
    \textgreater{} `pre-competitive' agricultural programs to support diversity
    \textgreater{} -- climate change adaptation, minor crop work, etc
  \item
    Continue funding public seed collections
  \end{itemize}
\end{itemize}
Strengthen the organic seed requirement
\begin{itemize}
\item
  Require organic growers to demonstrate improvement in organic seed
  \textgreater{} sourcing
  \begin{itemize}
  \item
    This was recommended by NOSB in 2018 -- has it been taken on?
    \textgreater{} Not by the time of the SOS 2022 report.
  \item
    For example, growers over a certain income threshold should
    \textgreater{} contract/take on research
  \end{itemize}
\item
  Improve availability resources
  \begin{itemize}
  \tightlist
  \item
    Measuring availability has been addressed in the EU through the
    \textgreater{} creation of an organic seed database, in which all organic
    \textgreater{} seeds must be registered
    \textgreater{} (\href{https://www.liveseed.eu/wp-content/uploads/2021/09/Policy-brief-RoadMap-National-Authorities_final_compressed-4.pdf}{{[}LiveSeed{]}}).
    \textgreater{} And consequently, these derogations are slated to be phased
    \textgreater{} out by 2036. The US has tried this but failed\ldots{}
  \end{itemize}
\item
  Increase certifiers enforcement ability
  \begin{itemize}
  \item
    First, focus on substantive non-compliance notices, rather than
    \textgreater{} administrative (Carter et al.~2022)
  \item
    Consider who exceptions are for: biggest farmers? Or most
    \textgreater{} diverse?
  \end{itemize}
\item
  Ultimately, goal is to consider tightening or closing the seed
  \textgreater{} loophole
\end{itemize}
The goal should be not to preference practical or principled,
completely. In order to give principled a chance of staying in it, there
should also be policies that challenge the highly consolidated nature of
seed in the US. Enforcing anti-trust and protecting access to plant
genetic material

In this way mixes are not just about utilizing a wide range of tools,
but also supporting a diversity of pathways towards compliance, without
preferencing a select farmer profile.

\hypertarget{mapping-the-spatial-boundaries-of-the-united-states-organic-seed-niche-an-empirical-test-of-the-global-innovation-system-framework}{%
\chapter{Mapping the spatial boundaries of the United State\textquotesingle s organic seed niche: An empirical test of the Global Innovation System framework}\label{mapping-the-spatial-boundaries-of-the-united-states-organic-seed-niche-an-empirical-test-of-the-global-innovation-system-framework}}

\chaptermark{spatial boundaries}

\hypertarget{introduction-2}{%
\section{1. Introduction}\label{introduction-2}}

Innovation systems (IS) are a complex network of actors and institutions
that interact in support of developing and diffusing new technological
and/or social innovation (Bergek et al.
2008). The field of IS has
broadly led to insights that support policy recommendations for
fostering innovation for sustainable development, particularly from
\textquotesingle niche\textquotesingle{} phases.

Yet, studies on innovation have long been grappling with the challenge
of defining meaningful boundaries around these innovation systems (Binz
et al.~2020; Coenen, Benneworth, and Truffer
2012). System boundaries
affect how we understand an innovation\textquotesingle s development and success
(Wieczorek et al.~2015),
and therefore identifying meaningful boundaries is a first step towards
a generalizable understanding of how and why systems evolve (Binz,
Truffer, and Coenen 2014).

Given the importance of boundary-setting, this paper asks: \emph{What
determines the spatial boundaries of an innovation system?} We answer
this question by drawing on the \textquotesingle Global Innovation Systems\textquotesingle{} (GIS)
framework (Binz and Truffer
2017), which proposes
predictors of IS spatial scales and sets out an approach for
operationalizing those predictors. The GIS framework outlines two
dimensions that help predict IS spatial structure: innovation \textquotesingle mode\textquotesingle{}
and how the innovation product is valued. Along these two dimensions, an
IS is operationalized into knowledge and valuation resource subsystems
to test the theory\textquotesingle s structural propositions. This paper aims to test
and deepen the framework\textquotesingle s theoretical underpinnings using a novel
empirical case and methodological approach.

We use the case of the organic seed system, a niche technological and
social innovation system where a variety of actors interact to support
development and use of organic seed. In accordance with the dimensions
of the GIS framework, this case combines a \textquotesingle doing-using-interactive\textquotesingle{}
innovation mode and customized product valuation, theoretically
positioning it as a \textquotesingle spatially sticky\textquotesingle{} regional innovation system.
Spatially sticky systems are those where knowledge and market related
resources are territorially embedded, making for closely knit regional
networks (Binz and Truffer
2017). We test this
proposition of the GIS framework using network data collected from
surveys of organic seed innovation system actors and analyze them using
two social network analytical approaches: n-clan analysis and
Exponential Random Graph Models.

Our results point to three main findings. First, the organic seed
networks show strong regional embeddedness in the structure of both the
knowledge and valuation resource subsystems, supporting the GIS
proposition that innovation mode and product valuation are strong
predictors of IS spatial boundaries (Binz and Truffer
2017)(\url{https://www.zotero.org/google-docs/?broken=laytXQ}). Second, our
findings suggest that higher-scale actors encourage vertical structural
coupling by acting as coordinating bridges between lower-scale actors
(MacKinnon, Afewerki, and Karlsen 2022; Rohe
2020). And third, the
knowledge and market based subsystems are horizontally coupled, where
activity in one subsystem positively affects activity in the other
(Tsouri, Hanson, and Normann
2021).

Altogether, these results confirm the expectations outlined by the GIS
and add nuance to the understanding of how actors link to one another
within and across spatial scales. Particularly, we show how even in a
regionally sticky innovation system, national and international-scale
actors still serve a prominent role in providing resources among
regional networks. Furthermore, actors tend to be active in more than
one resource subsystem, pointing to actors\textquotesingle{} ability to support multiple
functions and stimulate resource spillovers. The contributions made in
this paper are part of a broader effort to add theoretical depth to
innovation systems scholarship in order to outline generalizable
understanding of system boundaries (Kern 2015; Markard, Hekkert, and
Jacobsson 2015), and to
expand the field\textquotesingle s methodological toolkit (Köhler et al.
2019).

The paper will be structured as follows. In Section 2 we review existing
literature on the role of space in IS studies and outline the components
and dimensions of the GIS framework. We then describe the organic seed
system and its placement in the typology, and conclude the section with
an articulation of our hypotheses. In Section 3 we describe our research
methods, including our sampling approach and survey distribution,
network construction, and social network analysis methods. Next we
describe our results in Section 4, divided into descriptions of spatial
composition and inferences based on our network model. Last, we discuss
the results and their implications in Section 5 and conclude in Section
6.

\hypertarget{background-1}{%
\section{2. Background}\label{background-1}}

\hypertarget{the-spatial-turn-of-innovation-systems}{%
\subsection{2.1 The spatial turn of innovation systems}\label{the-spatial-turn-of-innovation-systems}}

The innovation systems (IS) perspective proposes that transitions are
driven by a broad range of networked interactions for developing,
promoting, and diffusing new technological and/or behavioral innovations
(Bergek et al.~2008).
Where to draw the boundary around those networked interactions has long
been of interest, resulting in several IS subfields, including national,
regional, technological and sectoral innovation systems (Carlsson et
al.~2002; Freeman 1995).
While these subfields impose boundaries around an IS, geography of
transitions scholars propose a spatial turn in order to think more
critically about boundary-setting (Binz et al.~2020; Coenen,
Benneworth, and Truffer 2012; Rohe
2020). Proponents of this
spatial turn argue that previous delineations often fail to take into
account heterogeneity at regional levels (Rohe and Mattes
2022), variation in
actors\textquotesingle{} roles depending on the scale at which they operate (Wieczorek
et al.~2015), and spillover
from global scales (Binz, Truffer, and Coenen
2014).

Scholarship in the geography of transitions focuses on the effects of
place and scale. Innovation systems are not homogenous across space, but
rather, there can be a great deal of regional, place-based variation
across broader innovation systems (Rohe and Mattes
2022). This heterogeneity
comes from innovation system functions that can be spatially \textquotesingle sticky\textquotesingle,
meaning they are closely linked to place (Rohe 2020; Rohe and Chlebna
2021). Beyond regional
heterogeneity, actors operate at different scales (e.g.~local producers,
regional organizations, national government actors, multinational
companies). The scale at which an actor operates influences both their
mandate and the resources they have to carry out that mandate (Bergek
et al.~2015). For example,
regional actors can be important bridges between local actors, as their
mandate is to connect different geographies across shared issues (F.
Vantaggiato et al.~2023).
And indeed, actor roles and scales can interact, as institutional logics
of different actors may shift as their scale of operation increases
(Lamers et al.~2017).
Furthermore, resources can spillover, both between regions (Kreft et
al.~2023) and at higher
scales through active national or international-scale actors (Wieczorek
et al.~2015).

To capture the effects of place and scale on innovation system
boundaries, Binz and Truffer (2017) develop a \textquotesingle Global Innovation
Systems\textquotesingle{} (GIS) framework to explain the spatial distribution of IS
resources. This framework proposes that an IS be operationalized as
subsystems based on its different resource-building relationships
(Binz, Truffer, and Coenen 2016; Musiolik, Markard, and Hekkert
2012). These subsystems are
then expected to match certain spatial characteristics based on two
conditions: innovation mode and type of product valuation. In the
following sections we describe the components for operationalizing an
innovation system network and the conditions that define the GIS spatial
typology.

\hypertarget{components-of-an-innovation-system-network}{%
\subsection{2.2 Components of an innovation system network}\label{components-of-an-innovation-system-network}}

Innovation systems are often defined in terms of functions and
structures (Wieczorek and Hekkert
2012). Functionally,
innovation systems rely on a series of processes that support its
development: entrepreneurial experimentation, knowledge creation,
collective influence on the direction of search, market formation,
resource mobilization, and creation of legitimacy (Bergek et al.~2008;
Hekkert et al.~2007). These
IS functions have been synthesized to represent \textquotesingle system resources\textquotesingle{}
provided by networks, summarized as knowledge, investment/capital,
market, and legitimation resources (Binz, Truffer, and Coenen
2016). Structurally, a
diverse set of actors, institutions and their interactions relative to a
technology are the scaffolding of an innovation system, responsible for
supporting these functions and building system resources (Jacobsson and
Johnson 2000; Musiolik and Markard
2011). These two features
of an IS, function and structure, map onto components of the GIS as a
network: \emph{innovation subsystems} and \emph{structural coupling} (Binz and
Truffer 2017).

The functions of an IS can be operationalized as innovation subsystems
(Binz and Truffer 2017).
These innovation subsystems are the networked relationships that arise
to carry out a particular function, such as developing and exchanging
knowledge, sharing resources, and creating legitimacy (Rohe
2020). There have been
several efforts to represent innovation subsystems based on specific
functional relationships. For instance, knowledge creation is often
operationalized by co-authorship (Binz, Truffer, and Coenen
2014), joint project
participation (van Alphen et al.~2010; Hermans et al.
2013), or co-patents
(Belderbos et al.~2014).
Other functions are more rarely quantified, such as legitimacy
(Heiberg, Binz, and Truffer 2020; Rohe and Chlebna
2021) and resource
mobilization (Giurca and Metz
2018). These networks of
functional relationships, which we will henceforth refer to as
\textquotesingle resource-based innovation subsystems\textquotesingle, provide a basis for
operationalizing the relationships that form the networks of an IS.

The connections that define the structure of an innovation subsystem are
made by actor-to-actor relationships. For each of these actors we can
identify certain spatial attributes: geographic place and/or operational
scale. How these actors create connections across space is
conceptualized as \textquotesingle structural coupling\textquotesingle{} (Binz and Truffer
2017). Under the initial
GIS framework, structural coupling is understood as the linkages within
or between scales of a subsystem that enable mobilizing of innovation
system resources. As this framework has developed, the structural
coupling concept has been extended to include different types: vertical
and horizontal (Rohe
2020).

Vertical coupling is a linkage between actors at different geographical
and/or operational scales. For example, actors from two different
regions may connect to help build legitimacy for an innovation
(MacKinnon, Afewerki, and Karlsen
2022), or a regional-scale
actor may connect with a national-scale actor to obtain resources like
knowledge or funding (Rohe
2020). The GIS emphasizes
this latter type of connection, where higher-scale actors \textquotesingle with global
reach\textquotesingle{} (Binz and Truffer 2017, p.1287) support lower-scale network
actors through resource provision (Bergek et al.
2015). However, the exact
understanding of how scale affects coupling within and/or between scales
of a subsystem still requires some clarity (Binz, Truffer, and Coenen
2014; Tsouri, Hanson, and Normann
2021).

Horizontal coupling is a linkage between resource subsystems, whereby
the resources developed in one network, such as knowledge development,
spillover into other networks to provide resources like legitimation and
market access (Rohe, 2020). For example, one study of Norwegian offshore
wind power finds that spillovers between resource subsystems depend on
the type and scales of knowledge creation activities, whereby
higher-scale (international) R\&D collaboration improves access to
international markets (Tsouri, Hanson, and Normann
2021). Together, different
forms of coupling link actors within and between places and scales, as
well as within and between resource subsystems, all of which create the
spatial architecture of the innovation system.

\hypertarget{typology-of-subsystem-spatial-configurations}{%
\subsection{2.3 Typology of subsystem spatial configurations}\label{typology-of-subsystem-spatial-configurations}}

Once innovation systems have been operationalized based on their
resource-based subsystems and structural couplings, the GIS framework
outlines dimensions that help predict when different spatial
configurations occur (Binz and Truffer
2017). Those two dimensions
are the innovation mode, related to the knowledge resource subsystem,
and product valuation, related to the valuation resource subsystem.

Innovation mode describes the types of knowledge and learning that are
required for the innovation to develop and thrive. The GIS defines two
end-points of an innovation mode gradient, ranging from \textquotesingle science and
technology\textquotesingle{} (ST) to \textquotesingle doing-using-interacting\textquotesingle{} (DUI) innovation
(Jensen et al.~2007).
These innovation modes align well with other conceptions of learning
across disciplines, such as codified vs.~tacit knowledge (Binz,
Truffer, and Coenen 2014),
and technical vs.~social and experiential learning (Lubell, Niles, and
Hoffman 2014). In the
former cases, ST knowledge is formalized, typically in science-based
industries, and can be transferred easily across contexts in forms like
patents and designs (Jensen et al.
2007). In the latter case,
DUI knowledge is created through practice and application, and is rooted
in social and experiential knowledge transfer (Gertler
2003). Regarding space, the
GIS proposes that innovations in the ST mode are less spatially bound
given knowledge\textquotesingle s codification and transferability, while DUI
innovations are more connected to specific places and contexts. The IS
resource relevant to this dimension is the knowledge resource, whereby
relationships for creating and diffusing knowledge form the knowledge
subsystem (Binz and Truffer
2020).

The product valuation dimension describes how a product is valued by
users (Binz and Truffer
2017). The GIS framework
identifies a gradient of valuation systems from standardized to
customized (Jeannerat and Kebir
2016). Standardized
valuation is where \textquotesingle consumption and legitimacy are stabilized around
clearly identified goods, services, and brands\textquotesingle{} (Binz and Truffer 2017,
p.1289). Products and services with customized valuation, in contrast,
build legitimacy and market-following based on the specialization of
their products to particular contexts (Jeannerat and Kebir
2016). This valuation
gradient aligns with different organizational types of value chains,
which contrast hierarchical, formalized and industrialized value chains
from democratic and territorially embedded value chains (Duncan and
Pascucci 2017; Gaitán-Cremaschi et al.
2018). Regarding space, the
GIS proposes that goods with standardized valuation can supply different
regional contexts without customization, and are therefore easily
transferable, where the same cannot be said for customized products. The
innovation system resources relevant to this dimension are the resource
mobilization, market formation, and legitimacy resources, whereby
material transfer, supply chain, and coalition building relationships
form the \textquotesingle valuation subsystem\textquotesingle{} (Binz and Truffer
2020).

Together, the innovation mode and product valuation dimensions of the
GIS create a two-by-two typology (Figure 2.1). This typology proposes
four types of innovation systems based on their relationship to space
and scale across their resource-based innovation subsystems. Innovation
systems in which the innovation is more ST-based and product valuation
is standardized are \textquotesingle spatially footloose\textquotesingle, in that they are likely to
have knowledge-related and valuation-related subsystems that have strong
spatial spillovers and therefore link at higher spatial scales (i.e.
\textquotesingle global\textquotesingle) (Figure 2.1. Quadrant I). At the opposite end, \textquotesingle spatially
sticky\textquotesingle{} systems represent a combination of DUI innovations and
customized product valuation, where the knowledge-related and
valuation-related subsystems are strong at the regional level (Figure
2.1. Quadrant III). The other two types mix these dimensions, whereby
\textquotesingle market-anchored\textquotesingle{} systems have spatially footloose knowledge-related
subsystems but sticky valuation-related subsystems, while
\textquotesingle production-anchored\textquotesingle{} spatial scales are reversed across subsystems.

\{width=``4.566288276465442in''
height=``3.834511154855643in''\}
\begin{figure}

{\centering \includegraphics[width=0.8\linewidth]{../osisn_spatial/figures/figure1} 

}

\caption{Global Innovation System typology, modified from Binz and Truffer 2017}\label{fig:unnamed-chunk-14}
\end{figure}
Of course, these are stylized types that omit other complex dimensions
(Binz and Truffer 2017).
For instance, different parts of an innovation\textquotesingle s value chain may exist
in different quadrants of the typology (Rohe
2020), knowledge creation
approaches may try to balance both ST and DUI modes (Jensen et al.
2007; Tsouri, Hanson, and Normann
2021), and placement on the
typology may change over time (Binz and Truffer
2017). However, this
framework sets out a baseline for theory building across several complex
dimensions, which is a contribution for developing a more generalizable
understanding of how and why systems evolve.

\hypertarget{organic-seed-innovation-system}{%
\subsection{2.4 Organic seed innovation system}\label{organic-seed-innovation-system}}

To test the GIS theory predicting the spatial boundaries of an
innovation system, we use the case of the organic seed system. Seed
systems represent technological (and in some cases social) innovations
for researching, producing, processing and selling seeds suitable for
agricultural production (Almekinders and Louwaars 2002; Lammerts van
Bueren et al.~2018). Seed
researchers develop and test new plant varieties through breeding
techniques, seed producers grow crops to harvest and sell their seed,
and seed processors and companies are the industries that process the
seed and sell it to consumers, from hobby gardeners to large scale crop
producers. Additionally, there are organizations and government agencies
that play several of these roles, including research, education, seed
saving, and certification.

The \emph{organic} seed system represents a niche that, at minimum, follows
certified organic standards throughout the innovation\textquotesingle s life cycle, and
at most, represents an alternative social-ecological approach to genetic
stewardship, land use, and community-building (Lammerts van Bueren et
al.~2018). We base our
study in the United States (US) organic seed system, but collect data so
as to open the boundaries of our system beyond national borders,
described in Section 3.1. In the US, certified organic crop production
is a niche system supporting less than 1\% of farmland (Bialik and
Walker 2019), nested within
the broader conventional agricultural innovation system (Rohe et al.
2022). Historically in the
US, the organic movement was borne out of a grassroots effort where the
values of soil health and localized food systems are generally shared
amongst the stakeholders (Youngberg and DeMuth
2013). This ethic has
extended to seed, where values of biodiversity stewardship,
decentralized ownership, and producer sovereignty are prominent among
organic seed stakeholders (Sievers-Glotzbach et al.~2020; Wood
2022).

The organic seed niche theoretically aligns with the dimensions of a
spatially sticky innovation system. While the knowledge to breed and
produce seed certainly requires technical skills, agricultural
production is a practice that is rooted to place. Furthermore, organic
seed breeding permits only the use of traditional breeding methods (e.g.
hybridization but not genetic modification) (Hubbard, Zystro, and Wood
2022), which places tacit
knowledge, or \textquotesingle doing-using-interacting\textquotesingle{} innovation, at the center of
plant breeding and production rather than science and technology.

Regarding valuation, the grassroots history of the organic movement has
generated a market environment where products are valued for their
regional adaptedness (Lammerts van Bueren et al.~2018; Rohe et al.
2022). Seeds that are
adapted to the climates, soil types, pest pressures, and cultural
histories of a given region are particularly important for organic
production, both for optimal productivity as well as regional
biodiversity maintenance (Hubbard, Zystro, and Wood
2022). This customized
approach contrasts from conventional seed and crop valuation, which aims
to create more spatially footloose products that can compete in a
standardized commodity market (Rohe et al.
2022).\footnote{System resources (also called \textquotesingle elements\textquotesingle) and IS functions
  generally map onto one another (Binz et al.~2016; Musiolik et al.
  2011). Musiolik et al.~2011 describe: ``System elements provide
  positive externalities such as public financial support, the
  deliberate diffusion of knowledge or the creation of legitimacy.
  These contributions at the system level can be allocated to the
  system functions and indicate how important the identified system
  elements for TIS development are.\textquotesingle{} (p.~1919 2011); And while some
  elements such as \textquotesingle value chain coordination\textquotesingle{} are not in the
  traditional list of functions (Bergek et al.~2008), in this paper we
  generally think of''functions'' and ``system resources'' as
  interchangeable, but try to primarily use the \textquotesingle resources\textquotesingle{} language
  over the \textquotesingle functional\textquotesingle{} language.}

\hypertarget{hypotheses-1}{%
\subsection{2.5 Hypotheses}\label{hypotheses-1}}

In this section we put forth hypotheses to test the Global Innovation
Systems framework using the case of the organic seed system. We propose
that the organic seed niche is positioned in the \textquotesingle spatially sticky\textquotesingle{}
quadrant of Figure 2.1, dominantly in the DUI innovation mode and
towards the center of the gradient in terms of product valuation. Based
on these dimensions, we expect that the organic seed innovation
subsystems\textquotesingle{} structures will generally align with the expectations of a
regional, spatially sticky GIS type proposed by Binz and Tuffer (2017).
Specifically, we hypothesize:
\begin{quote}
H1a: The knowledge-related resource subsystem will be strongly coupled
within regions, operationalized by regionally homogenous communities
and within-region homophily

H1b: The valuation-related resource subsystem will be strongly coupled
within regions, operationalized by regionally homogenous communities
and within-region homophily
\end{quote}
Beyond coupling within-regions, we also evaluate the importance of
vertical and horizontal structural coupling. While the GIS suggests that
spatially sticky systems structurally couple within and between regions
(Binz and Truffer 2020),
there is less clarity on how national actors connect across scales,
identified as the vertical coupling process. Thus we further the theory
by proposing that within a regionally sticky system, higher-scale actors
will provide both knowledge and market-based resources by acting as
coordinators (Bergek et al.~2015; F. Vantaggiato et al.
2023). Specifically, we
hypothesize:
\begin{quote}
H2: National and international-scale actors will be serve as bridges
between regional-scale actors, operationalized by national-actor
heterophily and the prevalence of national-actor network connections
\end{quote}
Further, we contribute to the developing examples of resource subsystem
spillovers via horizontal coupling (Musiolik and Markard 2011; Rohe
2020; Tsouri, Hanson, and Normann
2021) by hypothesizing:
\begin{quote}
H3: Actors\textquotesingle{} participation in one innovation subsystem will positively
influence participation in the other, operationalized by positive
correlation between the two subsystem networks and positive edge
covariates
\end{quote}
\hypertarget{methods-1}{%
\section{3. Methods}\label{methods-1}}

To test whether the organic seed network matches the expectations
outlined by the GIS spatial framework we rely on social network methods.
We operationalize the two innovation subsystems, knowledge and valuation
subsystems, based on network questions posed in surveys distributed to
stakeholders related to the organic seed innovation system. We then
summarize and analyze the two subsystems using two network approaches,
structurally descriptive summaries based on clan composition and
structurally explicit tests using Exponential Random Graph Models.

\hypertarget{organic-seed-system-surveys}{%
\subsection{3.1 Organic seed system surveys}\label{organic-seed-system-surveys}}

\hypertarget{innovation-system-stakeholder-identification}{%
\subsubsection{3.1.1 Innovation system stakeholder identification}\label{innovation-system-stakeholder-identification}}

We identified organic seed system stakeholder populations through two
processes: An initial compilation of several stakeholder databases and a
secondary snowball sampling process based on responses from the first
round of data collection. For the initial process, we generated three
databases with a total of 529 contacts. First, the USDA National Organic
Program INTEGRITY database was manually reviewed, with the support of
automated text methods, to identify 911 seed producers in the United
States, 390 (43\%) of which had valid email addresses and spatially
identifiable operations. Second, 87 organic seed companies, based
primarily but not entirely in the US, were identified by the national
non-profit Organic Seed Alliance, who keep a database for their
communication and research initiatives. Third, 52 organic seed
researchers based in the US were identified by reviewing seed-related
projects granted by major public and private organic grant funders (USDA
-SARE, -OREI, -NIFA, OFRF, and Ceres Trust) and Web of Science
publications over the last five years. This group included university,
governmental, and non-profit researchers. Note that our initial
population databases did not explicitly include governmental or
non-governmental organizations, apart from those that were related to
research.

For the second snowball process of identifying stakeholders, each survey
included network questions about respondents\textquotesingle{} connections in the seed
system (elaborated in Section 3.1.2). We used these responses to
identify a snowball sample, from which there were an additional 227 seed
system stakeholders identified, including 26 producers, 43 companies, 81
organizations, and 77 researchers (academic and governmental). While our
initial database creation focused on US actors, the snowball process
widened the scope of our respondents to include actors from other
geographies. In total, we identified 756 actors in the organic seed
innovation system, summarized by their geographic and scale
categorizations in Table 2.1. While setting boundaries around innovation
systems is a core challenge to the field (Bergek et al.
2008), we believe this
identification approach captures a wide range of both formal and
informal actors across multiple innovation system functions. And while
our population started with a national focus, the snowball sampling
approach accounted for the innovation system relationships that span
territorial boundaries.

\hypertarget{survey-development-distribution-and-sample}{%
\subsubsection{3.1.2 Survey development, distribution and sample}\label{survey-development-distribution-and-sample}}

We conducted a series of online surveys with the four groups of organic
seed stakeholders -- seed producers, companies, researchers (including
academics and government agencies), and organizations -- between
2020-2022. The surveys asked respondents about their roles in the seed
system and their networks: the people or organizations that they seek
and exchange information with, who they collaborate with on research,
who they work with along the supply chain, including seed contracts,
equipment rental, and sales, and who they source germplasm from for
breeding and exchanging seed.

Surveys were hosted on the Qualtrics survey platform and distributed
over email. Each potential respondent was sent an initial email
invitation with three reminders, spaced out every two to three weeks
(Dillman, Smyth, and Christian
2014). Seed producers,
researchers, and all snowball-sampled respondents were eligible for
survey completion incentives. In the case of seed producers, respondents
who completed the survey were put into a lottery system, in which 10
respondents were awarded \$100 Visa gift cards. For seed researchers,
each respondent earned a \$40 award in the form of cash, gift card, or
donation to an organization of their choosing. For the snowball sample,
each respondent earned a \$25 award in the form of cash, gift card, or
donation to an organization of their choosing.

Out of those we surveyed, we have responses from 247 actors across the
different roles for a combined response rate of 33\%. Based on data from
the survey and supplementary seed stakeholder databases, we are able to
add geographic and scale attributes to each actor that we surveyed.
Geographically, we group our US regional nodes according to the USDA's
Sustainable Agriculture Research and Education regions: West, North
Central, South, and Northeast. We explore the implications of this
regional boundary choice in Supplemental Material B-1. Actors from these
regions are represented as follows: 89 actors operating at the Western
regional scale, 48 at the North Central regional scale, 41 at the
Southern regional scale, and 27 at the Northeastern regional scale.
There are also 13 respondents that operate at regional levels across
Canada, which we group into four governmentally designated regions:
Pacific, Prairie, Central, and Atlantic (the fifth region, Yukon , was
not represented). We also have two respondents from the regional scale
in Mexico, listed under \textquotesingle Other country\textquotesingle.

For actors that span regional administrative or operational scales,
these were included in the analyses as \textquotesingle national\textquotesingle{} or \textquotesingle international\textquotesingle{}
actors. We assign these scales based on government agencies\textquotesingle{}
administrative role (e.g.~federal agency), self-described \textquotesingle national\textquotesingle{}
or \textquotesingle international\textquotesingle{} organizations, and size of company sales. Detail on
these scale designations is available in Supplemental Material B-2.
There were 23 survey respondents who operate at the national scale in
the United States, four that operate at the Canadian national scale, and
two that operate at the international scale.
\begin{table}

\caption{\label{tab:unnamed-chunk-15}Organic seed innovation system sample population and survey response rate by operational scale and geography}
\centering
\resizebox{\linewidth}{!}{
\begin{tabular}[t]{lrrrr}
\toprule
Region...scale & Population..N. & Sample..n. & Response.rate.... & Network.actors..n.\\
\midrule
West Regional & 334 & 89 & 27 & 194\\
North Central Regional & 143 & 48 & 34 & 120\\
South Regional & 94 & 41 & 44 & 87\\
Northeast Regional & 75 & 27 & 36 & 59\\
Pacific Regional & 17 & 6 & 35 & 13\\
\addlinespace
Central Regional & 8 & 3 & 38 & 7\\
Prairie Regional & 2 & 1 & 50 & 3\\
Atlantic Regional & 1 & 1 & 100 & 7\\
Other country Regional & 2 & 2 & 100 & 14\\
USA National & 64 & 23 & 36 & 88\\
\addlinespace
Canada National & 7 & 4 & 57 & 15\\
Other country National & 0 & 0 & NaN & 11\\
International & 9 & 2 & 22 & 27\\
Total & 756 & 247 & 33 & 645\\
\bottomrule
\end{tabular}}
\end{table}
\hypertarget{network-construction}{%
\subsection{3.2 Network construction}\label{network-construction}}

We constructed the innovation system network using the survey questions
related to actors\textquotesingle{} connections. These connections were elicited using
the \textquotesingle hybrid\textquotesingle{} network question approach (Henry, Lubell, and McCoy
2012). Using this approach,
respondents were asked to list up to five or ten people or organizations
(number varied with question) that they look to, work with, or source
from across several seed-related contexts. For each name they wrote,
respondents had the option to select the modes of each connection. We
match the types of connections to the functions described in the
innovation system functional framework (Bergek et al.
2008): knowledge creation,
resource mobilization, and market formation (Table 2.2). We then use
these connections to create our key units of analysis: the two
resource-based innovation subsystems as defined in the GIS framework
(Binz and Truffer 2017).
The knowledge subsystem includes connections related to information
acquisition and exchange, research partnerships, stakeholder involvement
in projects, and academic collaborations. The valuation subsystem
combines the resource mobilization and market formation functions, which
includes seed acquisition, seed exchange, research funding, licensing
agreements, contracts, renting, buying, and/or selling to/from. These
survey data do not include connection types that allow us to
operationalize the legitimation function.
\begin{table}

\caption{\label{tab:unnamed-chunk-16}Operationalization of innovation system resource-based subsystems based on relationship types}
\centering
\begin{tabular}[t]{lll}
\toprule
Functional.subsystems & Resource.subsystems & Types.of.relationships\\
\midrule
Knowledge & Knowledge creation & Information acquisition and exchange, Research partnerships, stakeholder project involvement, and academic collaborations\\
Valuation & Resource mobilization & Seed acquisition, seed exchange, research funding\\
Valuation & Market formation & Licensing agreements, contracts, rent, buy, and/or sell to/from\\
\bottomrule
\end{tabular}
\end{table}
Of the 247 survey respondents, 188 answered network questions, and these
respondents identified 497 other entities through 2395 connections.
Because 20\% (n=487) of the connections were to generic or
non-identifiable stakeholders (e.g.~respondents wrote in ``other farmers''
rather than ``Starlight Farm''), they were excluded from the network
structure (though the count of generic connections was included as a
control actor attribute). As a result, the final innovation system
network includes a total of 1908 ties between 645 uniquely identifiable
nodes across the different spatial scales (Table 2.1). For the sake of
our analysis, we consider a network tie to be non-directed, meaning we
assume the relationships go both ways, and we analyze each network
without any weighting of the connections, reducing our number of ties
from 1908 to 1206. For example, if actor A indicated multiple forms of
knowledge development with actor B (academic collaboration and research
project development), this counts as only one tie.

\hypertarget{social-network-analyses}{%
\subsection{3.3 Social network analyses}\label{social-network-analyses}}

We run two analyses to test whether the two resource-based subsystems in
the organic seed network are coupled within the regional scale (i.e.
\textquotesingle local\textquotesingle), within the higher scales (i.e.~\textquotesingle global\textquotesingle), and/or between
regional and higher scales. These analyses are a structurally
descriptive n-clan approach and a structurally explicit Exponential
Random Graph Modeling (ERGM) approach.

\hypertarget{geographic-and-scale-composition-across-n-clans}{%
\subsubsection{3.3.1 Geographic and scale composition across n-clans}\label{geographic-and-scale-composition-across-n-clans}}

We first describe our innovation network by summarizing the spatial
representation of actors across the two innovation subsystems by scale
and geography. We further this description by evaluating the composition
of spatial representation across clans in the subsystems. Clans are
cohesive subgroups within a network where the greatest distance between
any two nodes in the subgroup is no larger than the assigned path length
value (n) (Binz, Truffer, and Coenen 2014; Wasserman and Faust
1994). We identify clans
with a maximum path length value of 2 for each innovation subsystem to
detect small, cohesive units.

Once we identify the clans, we then categorize each of them based on
their spatial composition, which we determine based on actors\textquotesingle{}
geographic and scalar positions, guided by methods in Binz et al.
(2014). If the composition of a clan is dominated by a majority (\textgreater50\%)
of a particular geographic region or scale, that clan is assigned to
have either a \textquotesingle within-region\textquotesingle, \textquotesingle within-nation\textquotesingle, or
\textquotesingle within-international\textquotesingle{} composition. These are subgroups where the
majority of actors are within one specific geographic region or scale,
for example, Western regional or US national actors. If the composition
of the clan has no majority representation from a region but consists of
actors all of the same scale (e.g.~all regional actors split between the
West, North Central, and South regions or all national actors split
between US and Canada), the clan is assigned to a \textquotesingle between-region\textquotesingle{} or
\textquotesingle between-nation\textquotesingle{} composition. There is no option for a
\textquotesingle between-international\textquotesingle{} composition as international actors are not
assigned a geographic location. Last, if a clan has no dominant region
or dominant scale represented and mixes regional, national, and/or
international scales, it is assigned to a \textquotesingle mixed-scale\textquotesingle{} composition.

In short, the n-clan approach provides a cursory look at what geographic
and/or scale attributes dominate within small communities of the
subsystem. This analytical approach is considered structurally
descriptive, in that we utilize our network data to summarize spatial
information about our innovation subsystems (Scott and Ulibarri
2019). These n-clan
categorizations can suggest a trend in spatial structure, but this
method is limited in the ability to statistically test within or between
scale preference and prevalence.

\hypertarget{ergms}{%
\subsubsection{3.3.2 ERGMs}\label{ergms}}

To expand on the n-clan descriptive approach, we propose Exponential
Random Graph Models (ERGMs) (Lusher, Koskinen, and Robins
2012), a structurally
explicit way to statistically infer meaning from network structures
(Scott and Ulibarri 2019).
ERGMs are a well-developed network method (Lubell et al.
2012) that estimate the
likelihood that a network tie will form based on endogenous and
exogenous predictors (Robins, Lewis, and Wang
2012). Endogenous
predictors are the features of the network itself, such as the number of
connections and nodes, open and closed triangle formations, among many
others. Exogenous predictors are features of the nodes in the network,
which in the case of this study include an actors\textquotesingle{} spatial attributes.
An ERGM then estimates the likelihood that these features influence tie
formation in the observed network by simulating thousands of random
graphs that have the same structural parameters as those specified in
the model (Robins, Lewis, and Wang
2012).

To test our hypotheses regarding the spatial structure of the organic
seed innovation system, our results focus on two exogenous actor
attributes and one endogenous parameter . The exogenous attributes are
the geography and scale of actors in the network. To test whether there
is within region or within nation affinity in each subsystem (H1), we
use a \textquotesingle homophily\textquotesingle{} parameter for the combined geographic and scale
attributes. The homophily parameters in our model represent the
probability of an actor to preferentially connect to actors within their
own geography or scale. Second, to test the activity of actors operating
at certain scales (H2), we include a node covariate for the scale
attribute (regional, national, or international). These parameters
indicate the probability of forming a tie for an actor that operates at
the national or international scales compared to an actor that operates
at the regional scale. The endogenous parameter in our model is a matrix
representing the structure of the subsystem not being tested (e.g.~when
evaluating the knowledge subsystem the matrix represents the structure
of the valuation subsystem). This matrix, included in the model as an
edge covariate, allows us to test the strength of horizontal coupling
between resource subsystems (H3).

We also include seven endogenous structural terms as control variables.
The first three, edges, degree, and triadic closure, are common
structural parameters to account for social processes that take place in
the network (Lusher, Koskinen, and Robins
2012). Edges represent a
baseline probability a forming a tie, degree (specified with
geometrically weighting, \textquotesingle gwdegree\textquotesingle) represents the network\textquotesingle s
tendency towards centralization, and triadic closure (specified as
geometrically weighted edgewise shared partners, \textquotesingle gwesp\textquotesingle), represents
the likelihood of a connection to close a triangle (Levy and Lubell
2018).

The last four parameters are controls to account for the effect of the
data collection process on network sampling. These include a count of
the number of individuals that answered on behalf of one organization,
as these multi-respondent nodes had more chances to identify connections
compared to nodes with only one respondent. Second, we have a variable
representing the number of generic and/or non-identifiable ties an actor
had to account for actors\textquotesingle{} connectedness that we could not make
explicit in the network. Third, we include a binary variable
representing whether or not a stakeholder responded to the survey, as
non-respondents are likely to have fewer ties given that they could not
identify their relationships. Relatedly, our last control parameter is a
matrix of \textquotesingle structural zeroes\textquotesingle{} to specify that it was impossible for
two actors, neither of which responded to the survey, to have a
connection (Scott and Thomas
2015).

We estimate two ERGMs, one for each innovation subsystem, in order to
identify the variables that most influence network formation. All
analyses were conducted using R statistical software, relying primarily
on the `statnet` package (Handcock et al.
2019) for analysis. Code is
available at
\href{https://github.com/liza-wood/osisn_spatial}{{[}https://github.com/liza-wood/osisn\_spatial{]}}
and data is available upon request.

\hypertarget{results-1}{%
\section{4. Results}\label{results-1}}

\hypertarget{innovation-subsystem-and-n-clan-spatial-composition}{%
\subsection{4.1 Innovation subsystem and n-clan spatial composition}\label{innovation-subsystem-and-n-clan-spatial-composition}}

The organic seed innovation system identified through our survey methods
includes 645 stakeholders forming 1908 connections. Between the two
innovation subsystems, the knowledge subsystem is slightly larger, where
522 actors account for 1106 connections (815 when unweighted), while in
the valuation subsystem includes 418 actors accounting for 802
connections (643 when unweighted) (Table 2.3). Across the different
geographies, actors operating in different regions in the US account for
about 76\% and 69\% of the knowledge and valuation subsystems,
respectively. Actors operating at the US-national scale represent 12\%
and 16\% of the networks, international-scale actors account for 4\% and
3\% of the networks, and the remaining 8-12\% of actors are from various
regional and national scales. The majority of these are from Canada,
though there are also 17 actors from both scales represented as ``Other
country'' from various global contexts (Europe: 8, Africa: 3, Middle
East: 2, Latin America: 2, Asia: 2).
\begin{table}

\caption{\label{tab:unnamed-chunk-17}Actors' geographic representation of innovation subsystems by count and percent}
\centering
\begin{tabular}[t]{lll}
\toprule
Geography & Knowledge.subsystem & Valuation.subsystem\\
\midrule
West Regional & 160 (31\%) & 142 (34\%)\\
North Central Regional & 107 (20\%) & 75 (18\%)\\
South Regional & 76 (15\%) & 42 (10\%)\\
Northeast Regional & 51 (10\%) & 28 (7\%)\\
Pacific Regional & 6 (1\%) & 12 (3\%)\\
\addlinespace
Central Regional & 5 (1\%) & 4 (1\%)\\
Prairie Regional & 3 (1\%) & 1 (0\%)\\
Atlantic Regional & 3 (1\%) & 5 (1\%)\\
Other country Regional & 9 (2\%) & 10 (2\%)\\
USA National & 64 (12\%) & 66 (16\%)\\
\addlinespace
Canada National & 12 (2\%) & 9 (2\%)\\
Other country National & 3 (1\%) & 10 (2\%)\\
International & 23 (4\%) & 14 (3\%)\\
Total & 522 (NA\%) & 418 (NA\%)\\
\bottomrule
\end{tabular}
\end{table}
\hypertarget{ergm}{%
\subsection{4.2 ERGM}\label{ergm}}

Next we present results from our two Exponential Random Graph Models,
which express the likelihood of forming a tie in the innovation
subsystem based on actors\textquotesingle{} spatial attributes and activity across
subsystems. Figure 2.3 displays the coefficient estimates and confidence
intervals for the variables of interest in our two ERGMs, reported as
log-odds. Full model results are available in Appendix C-3. When
interpreting log-odds in the figure, a basic intuition is that values
less than zero represent lower probabilities of forming a tie and values
greater than zero represent higher probabilities of forming a tie. To
discuss specific statistics throughout the text, we transform the
log-odds coefficients into odds.

\hypertarget{organic-seed-as-a-spatially-sticky-regional-innovation-system}{%
\subsubsection{4.2.1 Organic seed as a spatially sticky regional innovation system}\label{organic-seed-as-a-spatially-sticky-regional-innovation-system}}

Network \textquotesingle homophily\textquotesingle, the tendency for actors with shared attributes to
form connections, can provide insight into our first hypothesis
regarding the spatial stickiness of the organic seed innovation
subsystems. The homophily terms in the model represent the estimated
likelihood of an actor from a certain region or scale (i.e.~four regions
in the US or operation at the US national scale) forming a tie with an
actor from that same region or scale. We report homophily only for
actors from the US regions and national scale because the number of
actors from other scales were too small to be estimated.

Across both innovation systems, actors from each region -- Northeast,
South, North Central, and West -- are all significantly more likely to
have connections to actors within their own region, rather than outside
it. When transforming the log-odds estimates for the knowledge
subsystem, actors from each respective region have odds of 16.5, 8.2,
9.2, and 3.3 to 1, all of which are quite large. When transforming the
log-odds estimates for the valuation subsystem, actors from each
respective region have odds of 14, 3.9, 2.8, and 4.3 to 1.

In short, actors that operate at a regional scale are highly likely to
connect to actors within their region, creating robust regional
networks. These results support H1a and H1b, as there is strong
within-region affinity within both subsystems, representing regional
stickiness of the organic seed innovation system. Furthermore, regional
affinity is slightly higher, on average, in the knowledge subsystem
compared to the valuation subsystem, supporting our placement of the
organic seed system more towards the center of the valuation axis
(Figure 2.1).

\{width=``6.036458880139983in''
height=``3.714744094488189in''\}
\begin{figure}

{\centering \includegraphics[width=1\linewidth]{../osisn_spatial/figures/figure3} 

}

\caption{Coefficient plot for ERGMs of the knowledge and valuation subsystems. Shapes represent the coefficient estimate, reported as log-odds, and bars represent confidence intervals.}\label{fig:unnamed-chunk-20}
\end{figure}
\hypertarget{vertical-coupling-via-national-scale-bridges}{%
\subsubsection{4.2.2 Vertical coupling via national-scale bridges}\label{vertical-coupling-via-national-scale-bridges}}

Network statistics about the affinity and activity of national and
international scale actors are relevant to our second hypothesis, as
they evaluate vertical coupling in the organic seed subsystems. The
national homophily term in Figure 2.3 demonstrates that actors who
operate at the US national scale are significantly less likely to
connect to one another in both the knowledge and valuation subsystems.
For both subsystems, the odds of a US national actor forming a tie with
another US national actor are small: 0.4 to 1. Thus in the organic seed
subsystems national actors are not interacting with other national
actors, but instead may be serving as bridges between regional-scale
actors.

This role of higher-scale actors as bridges is further supported by the
significant and positive estimates for tie formation at the national and
international scales. Where homophily represents the probability of a
connection to those with a shared attribute, the \textquotesingle scale\textquotesingle{} term in the
model represents the estimated likelihood of an actor to form a tie
generally, compared to the baseline of regional-scale actors. Compared
to this baseline, the odds for a national-scale actor to form a tie are
2.3 to 1 in the knowledge subsystem and 4.2 to 1 in the valuation
subsystem. For international actors, these odds are 3.8 to 1 and .9 to
1, respectively, though the valuation subsystem estimate is not
significant. These estimates indicate that ties to higher-scale national
and international actors are more likely than ties to regional-level
actors. These results support H2, suggesting vertical coupling via
higher-scale actors who are strongly linked to regional actors but not
to one another.

\hypertarget{horizontal-coupling-via-subsystem-spillover}{%
\subsubsection{4.2.3 Horizontal coupling via subsystem spillover}\label{horizontal-coupling-via-subsystem-spillover}}

Last, the activity of actors across both subsystems can provide insight
into our third hypothesis regarding horizontal coupling between organic
seed innovation subsystems. In other words, how network formation in one
resource subsystem affects the other. The \textquotesingle other subsystem structure\textquotesingle{}
term in the model represents how the presence of a connection in one
subsystem affects the likelihood of forming a connection in the other.
Consistent with H3, we find a large and significant effect of the other
subsystems\textquotesingle{} structure on the evaluated subsystem. Actors who have
connections in one subsystem are highly likely to form connections in
the other, with odds of 166 and 167 to 1 for the knowledge and valuation
subsystems, respectively.

\hypertarget{discussion}{%
\section{5. Discussion}\label{discussion}}

\hypertarget{developing-gis-theory}{%
\subsection{5.1 Developing GIS theory}\label{developing-gis-theory}}

Results from our analysis of the organic seed system align with the
expectations of the GIS in that both the knowledge and valuation
resource-based subsystems have strong within-region affinity (i.e.~they
are spatially sticky). By finding support for the GIS typology, this
paper contributes to the theory-building effort of the framework: to
identify the conditions under which innovation systems form and build
resources (Binz and Truffer
2017). This contribution is
part of a broader effort to add theoretical depth to innovation systems
scholarship in order to outline generalizable understanding of system
boundaries (Markard, Hekkert, and Jacobsson
2015).

This paper is among the handful of studies that have begun to test the
GIS across different empirical cases (MacKinnon, Afewerki, and Karlsen
2022; Rohe 2020; Tsouri, Hanson, and Normann
2021). The majority of
these studies relate to alternative energy (e.g.~wind power), which tend
to represent a mix of sticky and footloose spatial structure that rely,
to some extent, on strong global networks (ibid). And sustainability
transitions more generally tend to focus on innovations that rely more
on codified ST-based knowledge and products with more standardized
valuation (Köhler et al.
2019). Under the conditions
of a DUI innovation mode and customized valuation subsystem, as in the
organic seed case, we provide an important complement to existing
studies and highlight the importance of regionally-based networks (Rohe
and Mattes 2022).

The recognition that IS boundary-setting varies based on certain
conditions has important implications for the field. For innovation
studies researchers and policy makers alike, these conditions serve as a
guide on how to best select boundaries, which has important consequences
for measuring the development and success of an innovation system
(Binz, Truffer, and Coenen 2014; Wieczorek et al.
2015), as well as setting
effective policies.

Beyond validating the theory outlined by the GIS, we also add nuance to
the understanding of how subsystems link to one another within and
across space by testing the prevalence of both vertical and horizontal
coupling within the organic seed system (Rohe 2020; Tsouri, Hanson, and
Normann 2021). We find that
even in an IS rooted in strong regional networks, higher-scale actors at
the national and international scales have important roles to play as
bridges between regional networks. This result aligns generally with
expectations highlighted in the geography and innovation systems
literature (Binz, Truffer, and Coenen 2014; Wieczorek et al.
2015) and more broadly in
network governance studies (F. Vantaggiato et al.
2023).

Relevant to the vertical coupling results, we want to emphasize a
distinction between two types of vertical coupling. First, there is
coupling between geographies of the same scale, what MacKinnon et al.
(2022) refer to as trans-regional coupling, which we operationalize as
between-region clans and within-region heterophily. Second is coupling
between lower and higher scales, which we operationalize as mixed-scale
clans and infer from the national scale actors\textquotesingle{} simultaneous
heterophilic tendency (i.e.~preference to create ties with non-national
actors) and strong probability of tie formation compared to regional
actors. While we only observed the latter in our case, we propose that
researchers should continue to identify the different ways in which
actors\textquotesingle{} scales and roles affect the types of coupling they engage in,
and the different forms that structure can take (e.g.~supporting
coordination or cooperation (Berardo and Scholz
2010)).

Regarding horizontal coupling, the effect of one resource-based
subsystem on another, we find evidence for strong between-subsystem
spillover effects. In the organic seed system, actors\textquotesingle{} participation in
one innovation subsystem positively influenced participation in the
other, pointing to the multi-functionality of network relationships (F.
P. Vantaggiato and Lubell
2022). While our analysis
is not able to distinguish the effect of scale on spillover, which has
been relevant in other cases (Tsouri, Hanson, and Normann
2021), it is a first step
to quantify horizontal coupling using network methods.

\hypertarget{contributions-to-is-network-methods}{%
\subsection{5.2 Contributions to IS network methods}\label{contributions-to-is-network-methods}}

The analytical approaches used in this paper add to the methodological
toolkit of innovation research. While network analysis is certainly not
new to innovation studies ((van Alphen et al.~2010; Hermans et al.
2013; Musiolik, Markard, and Hekkert
2012), we employ an
Exponential Random Graph Modeling approach which is seldom applied to
this field (Hermans et al.
2017). We link the results
from the ERGM to a network approach previously applied to innovation
systems, n-clan composition (Binz, Truffer, and Coenen
2014), to provide a bridge
between descriptive methods (e.g.~(Giurca and Metz 2018; Rohe and
Chlebna 2022) and
structurally explicit methods (Scott and Ulibarri
2019).

The application of ERGMs in this paper heeds the call for new, rigorous
methods for analyzing innovation systems (Binz and Truffer 2017; Binz,
Truffer, and Coenen 2014)
and sustainability transitions more generally (Köhler et al.
2019). And indeed, it is
among other quantitative contributions for enriching the way we
understand relational data (Heiberg, Truffer, and Binz
2022), test theory, and it
is a complement to the studies that have analyzed GIS using qualitative
methods (MacKinnon, Afewerki, and Karlsen 2022; Rohe
2020).

\hypertarget{limitations}{%
\subsection{5.3 Limitations}\label{limitations}}

While the data collection approach used in this research aimed to
include all relevant geographies and scale, our methods certainly had
their shortcomings. First, our sampling approach allows us only to
discuss the spatial reach of the organic seed system \emph{based in the US}.
We used the US seed population as a starting point, then sampled to be
inclusive of all geographies and scales reported by the system\textquotesingle s
actors. Recognizing the starting point of our sample, we do not propose
this to be a global representation of the organic seed system. Rather,
we have focused on the US organic seed system and allowed snowball
sampling to define the boundaries according to the responses, which
permits us to evaluate the system without pre-specifying its limits.
Second, survey methods can be limiting. Surveys have the benefit of
actors self-identifying their connections, which may represent a more
genuine connection than codified relationships such as co-authorship or
joint project participation. The drawback, however, is that surveys are
challenging to collect, limited to the time of data collection, and
suffer from low response rates. As such we are capturing only a sample
of the innovation system population, from which we generalize to discuss
our findings. Future work should remain cautious about how the data
collection process may affect conclusions about system boundaries.

\hypertarget{conclusion}{%
\section{6. Conclusion}\label{conclusion}}

In this paper, we address a question of both academic and policy
relevance in innovation studies: \emph{What determines the spatial boundaries
of an innovation system?} We develop three hypotheses to test the theory
put forth by the Global Innovation Systems framework (Binz and Truffer
2017; Rohe 2020; Tsouri, Hanson, and Normann
2021). We use the case of
the organic seed system, a niche with a DUI innovation mode and
relatively customized type of product valuation, to validate the GIS
typology and extend our understanding of structural coupling within and
between innovation subsystems. By applying structurally explicitly
network analysis, Exponential Random Graph Models, we confirm the
expectations outlined by the GIS and add nuance to the understanding of
how actors link to one another within and across spatial scales.

This research contributes to the theoretical underpinnings of the GIS,
as well as adds to the methodological toolkit for analyzing innovation
systems using inferential network analysis.

In practical terms, understanding the spatial boundaries most relevant
to an innovation system can help policymakers identify the scope of
relevant policy. In organic seed, for example, recognizing its spatial
stickiness highlights the importance of supporting regional seed
policies, for example state-specific protections for genetic
contamination, to help meet the customized needs of an area (Organic
Seed Alliance 2022). Though
at the same time, policy should leverage the bridging role of
higher-level national actors as connectors of limiting resources like
funding and providing pre-competitive resources like genetic material.

\hypertarget{innovation-system-network-formation-resource-constellations-shape-cooperation-in-the-organic-seed-niche}{%
\chapter{Innovation system network formation: Resource constellations shape cooperation in the organic seed niche}\label{innovation-system-network-formation-resource-constellations-shape-cooperation-in-the-organic-seed-niche}}

\chaptermark{network formation}

\hypertarget{introduction-3}{%
\section{1. Introduction}\label{introduction-3}}

How to build a successful innovation system has long been a topic in the
sustainability transitions literature (Hermans, Klerkx, and Roep 2015;
Klein Woolthuis, Lankhuizen, and Gilsing
2005). Much attention has
been paid to the functions that support innovation systems, such as
knowledge creation, legitimacy building, capital support, and market
formation, which are the processes fueling the development and diffusion
of an innovation (Bergek et al.~2008). Complementary to these functions
are structural features, which define the architecture that helps these
functions develop, accrue, and/or dissolve (Weiczorek et al 2012).

While it is broadly agreed that there is no single recipe for how
innovation systems are built (Bergek et al.~2008), recent attention has
been paid to innovation systems\textquotesingle{} structural features and the conditions
that predict them (Binz and Truffer 2017; Binz et al.~2014; Hermans
2013). Of particular interest in this paper is the resource-based theory
for system building (Musiolik et al.~2020, Musiolik et al.~2012), where
the availability and distribution of resources (i.e.~resource
constellations) predict the types of structures that network actors
form. Yet, empirically testing this theory is still in early phases
(Cholez and Marie-Benois 2023), pointing to an opportunity to examine
and extend the theoretical underpinnings of network formation and
actors\textquotesingle{} role in building innovation systems.

In this paper we ask: \emph{What are the conditions driving innovation system
formation? And what types of actors are most involved in the formation
of innovation networks?} We connect the resource-based theory for
network building (Musiolik et al.~2020) to governance and social network
literature (Provan and Kenis 2008, Prell?) to develop a testable
hypothesis about the relationship between system resources and network
structure. Specifically, we test whether \textquotesingle partner modes\textquotesingle{} of system
building, represented by closed triangle structures in a network, will
be more prevalent than \textquotesingle intermediary modes\textquotesingle, represented by open
triangles, in innovation system networks that have existing,
decentralized resources. We then draw on the multi-actor perspective
(Avelino and Wittmayer) to extend this theory and hypothesize how
actors\textquotesingle{} institutional logics influence their activity across different
subsystems in the network. Specifically, we propose that non-profit
stakeholders will be more active in the subsystems formed to generate
knowledge resources, and for-profit stakeholders will be more active in
subsystems formed around valuation resources like mobilizing funding,
value chain creation, and market formation.

We test these hypotheses using the empirical case of the United States
organic seed niche innovation system. We construct an innovation system
network of over 800 actors using survey data collected between
2020-2022, and operationalize two resource subsystems -- knowledge and
valuation subsystems -- based on various relationships between actors.
Based on these data, we use a structurally explicit network analytical
approach, exponential random graph modeling, to statistically determine
whether certain network-building processes and actor compositions are
more or less prevalent in the organic seed niche\textquotesingle s resource subsystems.

Our results point to two main findings. First, we provide empirical
support for the resource-based theory, which predicts innovation system
structure based on resource constellations. Both the knowledge and
valuation resource subsystems in the organic seed system have existing
but decentralized resources, and these networks rely significantly more
on \textquotesingle partner modes\textquotesingle{} of system building (closed triangle structures)
than \textquotesingle intermediary modes\textquotesingle{} (open triangles). Second, we find support
for our hypotheses about actor composition, where actors\textquotesingle{} institutional
logics align with their relative activity in the different resource
subsystems. Apart from government, non-profit actors are more active in
subsystems related to knowledge resources, aligning with pre-competitive
logics, while for-profit actors are more active in subsystems related to
valuation resources, aligning with more competitive logics

Altogether, these results confirm the expectations outlined by the
resource-based theory for system building and add nuance to the
understanding of when certain actors are more or less likely to be
involved. The contributions made in this paper align with the interest
of Musiolik et al, (2012), who claim that ``in order to better
understand and improve system performance, we have to look into the
processes through which technological innovation systems are
intentionally created'' (p.~1034). And by furthering our understanding
about the building blocks of innovation systems, these findings can help
policymakers better understand the circumstances under which certain
system building strategies should be put into place.

The paper will be structured as follows. In Section 2 we review existing
literature on the structural building blocks of networks and the
resource and actor-based attributes that theoretically shape innovation
system structure, based on which we articulate our hypotheses. We then
describe our case of the organic seed system and the kinds of networks
it includes. In Section 3 we describe our research methods, including
our data collection, network construction, and social network analysis
methods. Next we describe our results in Section 4, which include
descriptions of organic seed system resources and the resource
subsystems, and inferences about network structure and composition.
Last, we discuss the results and their implications in Section 5 and
conclude in Section 6.

\hypertarget{background-2}{%
\section{2. Background}\label{background-2}}

\hypertarget{the-structural-building-blocks-of-networks}{%
\subsection{2.1 The structural building blocks of networks}\label{the-structural-building-blocks-of-networks}}

\hypertarget{modes-of-innovation-system-building}{%
\subsubsection{2.1.1 Modes of innovation system building}\label{modes-of-innovation-system-building}}

How social and technological innovations develop and diffuse is of
central interest to innovation system (IS) research. The functional and
resource-based frameworks of innovation systems (Bergek et al.~2008;
Musiolik et al.~2012) have been influential in delineating the processes
that foster an innovation system\textquotesingle s formation. For example, functions
like knowledge development and diffusion, legitimation of the
technology, resource mobilization, and market formation provide key
resources necessary to support an innovation\textquotesingle s success ({[}Binz et al.,
2016{]}). The functional
perspective initially argued that the structure of an innovation system
-- who is involved and how they are connected -- is an emergent property
for which there was no single recipe (Bergek et al.~2008). However,
scholars argue that this approach removes agency from IS studies (Farla
et al.~2012;; Musiolik et al.~2012), which has in turn minimized
attention placed on the structural building blocks of innovation
systems.

There has been renewed attention on the structure of innovation system
formation (Musiolik et al.~2020; Binz and Truffer 2017). One key example
is how actors can employ different modes of \textquotesingle strategic system
building\textquotesingle{} (Musiolik 2020; Cholez and X 2023). System building is the
creation and strengthening of innovation system-wide resources,
including functions\footnote{System resources (also called \textquotesingle elements\textquotesingle) and IS functions
  generally map onto one another (Binz et al.~2016; Musiolik et al.
  2011). Musiolik et al.~2011 describe: ``System elements provide
  positive externalities such as public financial support, the
  deliberate diffusion of knowledge or the creation of legitimacy.
  These contributions at the system level can be allocated to the
  system functions and indicate how important the identified system
  elements for TIS development are.\textquotesingle{} (p.~1919 2011); And while some
  elements such as \textquotesingle value chain coordination\textquotesingle{} are not in the
  traditional list of functions (Bergek et al.~2008), in this paper we
  generally think of''functions'' and ``system resources'' as
  interchangeable, but try to primarily use the \textquotesingle resources\textquotesingle{} language
  over the \textquotesingle functional\textquotesingle{} language.} like shared knowledge, legitimation, market
formation, and value-chain creation (Musiolik et al.~2011; Musiolik et
al.~2012, Hellsmark and Jacobbson 2009). Recent research has focused on
how these system resources are intentionally developed by building
relationships, specifically within formal networks (Musiolik and Markard
2011).

Systems building theory lays out two types of multi-actor structures
within innovation system networks that can support system resources. The
first is \textquotesingle partner mode,\textquotesingle{} where actors in the network share
complementary skills and assets to co-create resources that support
system-level functioning. The second is \textquotesingle intermediary mode\textquotesingle, where
actors work together via an intermediary actor that can generate
resources that support both the network, such as trust and reputation,
as well as systems-level functioning (Muisolik et al.~2020). The two
structures are displayed in Figure 3.1A. The key structural difference
is that the intermediary mode structure requires the creation of network
resources such as trust and reputation, and so \textquotesingle a system builder
coordinates activities in a network of collaborators to develop new,
intermediate organizational structures\textquotesingle{} (p.10, Musiolik et al.~2020).
We use these structures as a starting point for linking the innovation
systems and social networks literatures.

\{width=``6.5in'' height=``3.3333333333333335in''\}
\begin{figure}

{\centering \includegraphics[width=1\linewidth]{../osisn_processes/figures/figure1} 

}

\caption{A) Multi-actor network structures, adapted from Musiolik et al. (2020). Partner mode structures (top) and intermediate mode structures (bottom) show how system builders connect to other actors to build system resources (SR) and network resources (NR). Circles represent different actors, labeled as system builders (SB) or complementary actors (A). B) Social network motifs, closed triangles (top) and open triangles (bottom), which align with the modes in panel A. C) Examples of two closed network structures, each representing relationships formed in different resource subsystems, the knowledge subsystem (top, blue solid connecting lines) and the valuation subsystem (bottom, green dashed connecting lines). Actor composition varies based on subsystem, with different combinations of universities (Uni), organizations (Org), government (Govt), companies (Co), and producers (Pro).}\label{fig:unnamed-chunk-21}
\end{figure}
\hypertarget{social-network-motifs}{%
\subsubsection{2.1.2 Social network motifs}\label{social-network-motifs}}

Theoretical and methodological traditions in social network analysis can
help further our understanding of network structuring modes studied in
innovation systems. While a growing number of studies operationalize
innovation systems as networks (van Alphen 2010, Fernandez 2020, Giurca
and Metz 2018; Lopez Hernandez 2019; Binz et al.~2014, Rohe and Chlebna
2022), few connect to the theory on network structure (Hermans et al.
2013b; Hermans et al., 2017).

Studies of social networks have long recognized that networks are made
up of structural building blocks (Lusher et al.~2012). These structural
building blocks, also called motifs, are the quantifiable foundations of
a network that researchers use to develop and test social network
theories (Granovetter 1973, Burt 2000; Bodin and Tengo 2012). For
example, two motifs that have been central in the recent studies of
governance networks are open and closed triangles ( Berardo and Scholz
2010; Vantaggiato) (Figure 3.1B). These motifs have explicit structural
meaning and come by many names (Prell and Skorovitz 2008 ; Levy and
Lubell 2018). Open triangles depict one actor connecting two other
actors who are otherwise not connected to one another, also referred to
as brokerage (Burt 2005) or coordination (Berardo and Scholz 2010). This
open triangle configuration, measured by a network\textquotesingle s
\textquotesingle centralization\textquotesingle, is associated with efficiency and reduced
transaction costs, as one actor takes on a centralized leadership role
(Burt 2000). On the other hand, a closed triangle depicts three actors
who are all connected to one another, also referred to as closure (Burt
2005) or cooperation (Berardo and Scholz 2010). This cooperation
configuration, measured by a network\textquotesingle s \textquotesingle transitivity\textquotesingle/\textquotesingle triadic
closure\textquotesingle, suggests co-development and exchange of ideas, as well as
trust-building and cohesion (Burt 2005, Prell and Lo 2016). The exact
social meaning behind these structures is contested (Prell) and they are
not mutually exclusive (Vantaggiato, Levy and Lubell 2018), but still
they serve as foundational motifs around which to build and test theory.

We propose that the partner and intermediary modes put forth by Musiolik
et al.~(2020) map onto the closed and open triangle motifs used in
social network research. The partner mode of system building is akin to
closed triangles, where partnerships are primarily about cooperating to
share complementary resources (Musiolik et al.~2020). The intermediary
mode of system building is akin to open triangles, where the
systems-level deficits are more complex and network-level resources like
coordination are required to support building up of system resources
(Musiolik and Markard 2012; Musiolik et al.~2020). A note is that though
the partner mode configurations all structurally look like open
triangles in their original conception (Figure 3.1A), we argue that the
theoretical underpinnings of this structure align with those of closed
triangles.\footnote{We believe that the structures observed by Musiolik et al.~(2020)
  are limited based on the authors\textquotesingle{} \textquotesingle ego-centric\textquotesingle{} interview methods
  (Chung et al 2005) which relied largely on the perspective of key
  actors to describe how networks are formed by system builders only.
  A comprehensive sampling method that gathers data on all actors\textquotesingle{}
  connections would be able to more fully represent the network
  structures and likely show \textquotesingle closure\textquotesingle{} of triangles in the cases of
  partner mode structures.}

By linking theories in innovation systems to social networks, we can use
social network methods to test and extend our understanding of
innovation system formation. While several studies have recently used
networks to \emph{describe} innovation systems (e.g.~van Alphen 2010, Giurca
and Metz 2018; Binz et al.~2014, Rohe and Chlebna 2022), only small
handful of innovation systems papers have employed \emph{inferential} network
analysis, such as testing for network-level cooperation and identifying
prominent types of actors (Hermans 2013b, Hermans 2017). We propose that
inferential network analysis, which statistically tests for the
prevalence of different structures and other network attributes compared
to randomly generated networks with similar features (Scott and Ulibarri
2019), can help us test and advance existing innovation system theory
regarding networks.

\hypertarget{attributes-shaping-innovation-system-network-structure}{%
\subsection{2.2 Attributes shaping innovation system network structure}\label{attributes-shaping-innovation-system-network-structure}}

\hypertarget{resource-based-conditions-across-subsystems}{%
\subsubsection{2.2.1 Resource-based conditions across subsystems}\label{resource-based-conditions-across-subsystems}}

In this paper we contribute to a central question in innovation system
network research: What are the conditions driving innovation system
formation? Recent theoretical developments identify several conditions
predicting different structural configurations (Musiolik et al.~2020;
Binz and Truffer 2017). To describe system building, Musiolik et al.
(2020) put forth a resource-based approach to predict whether
multi-actor network structures will pursue partner or intermediary
modes. This approach proposes that actors pursue different structures
based on resource availability and resource concentration within the
system, also called \textquotesingle resource constellations.\textquotesingle{} If the resources (e.g.
skills or assets) needed to support a systems-level resource (e.g.
knowledge diffusion and market formation functions) already exist but
are dispersed among different actors, a partnering structure is more
likely. On the other hand, if the resources are non-existent in the
innovation system, the intermediary mode of structuring is more likely
to support the creation of new network and systems-level resources
(Musiolik et al.~2020).

The resource-based reasoning supporting the innovation system building
theory resembles the logic of other social network theories,
particularly those from network governance (Provan and Kenis 2008).
Network governance theory proposes that network structure is dependent
on a handful of contingencies, one of which is the \textquotesingle need for
network-level competencies\textquotesingle{} -- a concept closely aligned to network
resources like trust, legitimacy, and coordination to address complex
problems (Provan and Kenis 2008; Lubell et al.~2017). When the need for
these network-level competencies is high, networks will be coordinated
by a lead or external agency with high centralization (open triangles)
rather than cooperation (closed triangles) (Lubell).. On the other hand,
when the need for network-level competencies is low, networks will take
a participant-led approach where actors partner to co-create
complementary alliances (Rudnick).\footnote{We do note, however, that some policy network theory proposes the
  opposite relationship between network structure and the need for
  network-level competencies like trust and addressing complex
  problems. Specifically, Berardo and Scholz (2010) propose that when
  the challenges facing the network have low complexity and therefore
  require fewer resources and low-risk, then centralized,
  open-triangle structures prevail. As complexity increases and there
  is more of a need for creative solutions, transitive,
  closed-triangle structures prevail. This contrasting view represents
  the ongoing debate in network studies on the social processes
  representing by the classic open and closed triangle motifs.} These network governance concepts
map well onto Musiolik et al\textquotesingle s intermediate and partner modes (2020),
and complement them by aligning them with explicit structural motifs.

Testing these theories can be challenging because innovation systems are
a complex web of several overlapping networks. Networks are
multi-functional (Vantaggiato and Lubell 2022) and actors often engage
in several different kinds of relationships that address different
resource needs. Networks can help actors create and exchange knowledge,
develop contracts, mobilize resources, and/or connect along the value
chain (Spielman et al.~2011; Giurca and Metz 2018; Truffer and Binz
2017). The functional purposes of a network connection are not
necessarily discrete, especially as the connections become repeated over
long-term interaction (Oh et al.~2004, BETTER CITE). For example,
agricultural production contracts were designed to support market
development and resource mobilization within the value chain, but as a
byproduct they also support knowledge development and exchange (Cholez
2020).

To capture the complexity of innovation networks, we conceptualize
innovation systems as a series of interdependent \textquotesingle resource
subsystems.\textquotesingle{} We follow the typology developed by Binz and Truffer
(2017), recognizing a knowledge subsystem, composed of knowledge
development and diffusion functions, and a valuation subsystem, composed
of resource mobilization, market formation, and value chain coordination
functions (Musiolik and Markard 2011). These knowledge and valuation
resource subsystems represent networks -- both formal and informal --
that involve joint activities directly linked to building innovation
system resources. These resource subnetworks are like layers in the
total innovation system network, where the structure of each builds the
structure of the whole (Figure 3.2).

\{width=``6.5in'' height=``2.2745975503062117in''\}
\begin{figure}

{\centering \includegraphics[width=1\linewidth]{../osisn_processes/figures/figure2} 

}

\caption{A) (left) A subsystem/network to address one kind of system resource deficit (e.g. knowledge development). B) (center) A subsystem to address a second kind of system resource deficit (e.g. market formation). C) (right) The combination of both resource subsystems, which make up a multi-functional innovation system network.}\label{fig:unnamed-chunk-22}
\end{figure}
We use these two resource subsystems -- knowledge subsystem and
valuation subsystem -- as cases for testing the resource-based theory.
\begin{quote}
H1: Partner mode network structures, represented by closed triangles,
will be more prevalent than intermediary mode network structures,
represented by open triangles, when innovation resource subsystems
have existing, decentralized resources
\end{quote}
\hypertarget{actor-composition-based-on-institutional-logic}{%
\subsubsection{2.2.2 Actor composition based on institutional logic}\label{actor-composition-based-on-institutional-logic}}

Actor agency is central to innovation system formation (Musiolik and
Markard 2011; Kern 2015). For example, \textquotesingle system builders\textquotesingle{} are prime
movers who create cohesion and direction within an innovation system
(Hellsmark and Jacobsson 2009; Musiolik et al.~2020). Similarly,
\textquotesingle intermediary\textquotesingle{} actors come in several forms to help facilitate
transitions (Kivimaa et al.~2019). However, there is still a limited
understanding of how different actor types participate in the innovation
system and their roles in building different system resources (Hermans
et al.~2013, Van Alphen 2010; Binz et al.~2014, Lamers et al.~2017).
Thus we turn to our second research question: What types of actors are
most involved in the formation of innovation networks?

We rely on the multi-actor perspective as a guide to classifying actors
and understanding their activity within different resource subsystems
(Avelino and Wittmayer 2016).\footnote{We choose to focus on actor categories rather than the existing
  classification of \textquotesingle system builders\textquotesingle{} (Musiolik et al.~2020) for
  three reasons. First, classifying system builders seems to have
  little formula: when and how an actor is classified as a system
  builder appears largely a matter of perspective. In line with
  challenges in multi-stakeholder assessment, classifying actors into
  bins such as key players ``tends to identify the `usual suspects'
  and there is a danger that this may lead to the under-representation
  of marginalised or powerless groups'' (Reed et al.~2009). Second,
  system building is largely relational, and therefore this dynamic of
  coordinating and/or bridging will still come through in the results
  based on our network approach, allowing us to still include system
  building in our discussion. Third, we argue that the line between
  strategic and emergent system formation is blurred. While system
  building is guided by individuals it relies on the whole network and
  is therefore a collective process (Musiolik et al.~2020), making the
  \textquotesingle builder\textquotesingle{} versus \textquotesingle other actor\textquotesingle{} distinction less important. We
  make this point further with regards to formal and informal networks
  in our case description (Section 3).} This approach sets out four categories
of actors: state (public agencies and researchers), market (private
firms), communities (households, families), and the third-sector
(non-profit organizations). Hybrid categories can also emerge at the
intersection of these types, such as public-private partnerships. Among
these categories, actors (which can be identified as individuals or as
part of a broader organizational structure) are distinguished by their
agency (e.g.~a governmental researcher). In contrast, the institutions
to which actors belong (e.g.~public institution) do not have agency but
rather they align with a particular \textquotesingle logic\textquotesingle{} (Avelino and Wittmayer
2016). For example, public agencies are guided by their social
commitment, firms are guided by efficiency, and third-sector
organizations are guided by a blend of these motivations. Similar
concepts are described in the Triple Helix model\textquotesingle s proposal that
university, industry, and government each have their own selection
pressures: novelty production, wealth creation, and normative control,
respectively (Leydesdorff and Meyer 2006). These perspectives help guide
our expectations of what kinds of actors we\textquotesingle re likely to see involved
in innovation resource subsystems and why.

We propose that there will be variation in the prevalence of different
actor-types across the different resource subsystems (Figure 3.1C).
Actors from non-profit sectors such as government agencies, public
research institutions, and non-profit organizations are traditionally
guided by social commitment (Avelino and Wittmayer 2016), and generation
of pre-competitive public goods such as knowledge resources (King et al.
2012). Theory on strategic niche development reinforces this point given
that governmental support is often considered important for protecting
and nurturing niche innovation spaces before they are profitable (Smith
and Raven 2012). Examples from the privatization of agricultural
extension also strengthen this dynamic, finding that private control of
knowledge development and outreach tend to fall short to publicly
managed models (Eastwood et al.~2017; Labarthe and Laurent, 2013).
Further, explorations of knowledge creation networks find that actors
from universities and research institutes account for the vast majority
of actors compared to those from for-profit groups (van Alphen 2010;
Binz et al.~2014).
\begin{quote}
H2a. Non-profit stakeholders (government, university, and
organizations) are more active in the knowledge resource subsystem
compared to other actors.
\end{quote}
On the other hand, for-profit actors typically engage in competitive and
profitable activities (King et al.~2012), and so processes like resource
acquisition and developing business relationships better align with
their incentive structures. For example, research on innovation
platforms finds that there is a diverse representation of groups in the
knowledge development and diffusion stages, but this representation
shifts towards a higher prevalence of private actors in the market
formation stage (Lamers et al., 2017).
\begin{quote}
H2b. For-profit stakeholders (companies) are more active in valuation
resource subsystems compared to other actors.
\end{quote}
\hypertarget{case-organic-seed-innovation-system}{%
\subsection{2.3. Case: Organic seed innovation system}\label{case-organic-seed-innovation-system}}

We test our hypotheses about resource subsystem formation using the case
of the organic seed system in the United States -- a niche innovation
system linked to the broader organic agriculture niche. Organic seed
systems represent technological (and in some cases behavioral and
social) innovations for researching/breeding, producing, processing and
selling seeds suitable for organic agricultural production (Almekinders
and Louwaars 2002). Seed researchers develop and test new plant
varieties (ideally those suitable for organic conditions, Rohe forever
niche), seed producers grow crops under organic conditions to harvest
and sell their seed, and seed processors and companies are the
industries that process the seed and sell it to consumers, from hobby
gardeners to large scale organic crop producers. Additionally, there are
organizations and government agencies that play several of these roles,
including research, education, seed saving, and certification. More
details about the history of the organic seed system in the US are
described in Chapter 1 and Chapter 2, as well as in Rohe and Seievers
for a European context.

The organic seed innovation system includes a variety of networks that
serve different functional purposes, from seed exchange and knowledge
diffusion to equipment rental and sales. These networks are a blend of
both formal and informal networks. A formal network is an
\textquotesingle organizational structure with clearly identifiable members where firms
and other organizations come together to achieve common aims\textquotesingle{} (p.~1034,
Musiolik et al.~2012). There are clear examples of such networks in the
organic seed system, such as membership in a grower\textquotesingle s association,
contracting with a seed processor, and collaboration in joint research
projects. There are also various informal networks, such as knowledge
exchange with neighboring farmers and exchanging seed at seed swaps. The
categorization of several other networks, however, is less clear; for
example, subscription to a listserv, attending a conference, or renting
equipment. All of these connections are ways that stakeholders develop
the innovation and shape the system, but membership can be transient
(e.g.~conference attendees), and whether this is strategic or emergent
is hard to delineate (Musiolik et al.~2020).

Recent theory has focused on formal networks, given that they are venues
for strategic system building (Musiolik et al.~2020; Cholez and
Marie-Benoit 2023). However, we include both formal and informal
networks in this study for two reasons. First, the practical line
between formal and informal networks is blurry. Formal networks may
generate informal networks (Klerkx and Proctor; Cholez and Marie-Benoit
2023), informal exchanges may at some point become formalized (CITE),
and indeed, informality has been found to have a stronger influence of
beliefs and behaviors (Prell et al.~2010). Second, acknowledging the
gradient of network (in)formality more fully represents the complexity
of innovation systems, where emergent and strategic system formation are
not mutually exclusive (Van de Ven, Musiolik et al, 2020). Thus to
capture the full range of relationships that help form a network in the
organic seed systems, we consider the wide range of relationships that
support different system resources.

\hypertarget{methods-2}{%
\section{3. Methods}\label{methods-2}}

\hypertarget{data-collection}{%
\subsection{3.1 Data collection}\label{data-collection}}

We identified organic seed system stakeholder populations first through
comprehensive database creation of seed producers, companies, and
researchers, followed by a snowball wave of survey sampling that helped
us identify additional actors and relevant non-governmental
organizations. Greater detail of this sampling process is in Chapter 2
(Section 3.1.1). In total, we identified a population of 756
stakeholders in the organic seed innovation system, summarized in Table
3.1. While setting boundaries around innovation systems is a core
challenge to the field (Bergek et al.~2008), we believe this
identification approach captures a wide range of both formal and
informal actors across multiple innovation system functions.

We conducted a series of online surveys with the four groups of organic
seed stakeholders -- seed producers, companies, researchers (including
academics and government agencies), and organizations -- between
2020-2022. The surveys asked respondents about their role in the seed
system, their perceived challenges and expertise, and perceptions on
several issues salient in organic seed. Additionally, respondents were
asked about the people or organizations that they seek and exchange
information with, who they collaborate with on research, who they work
with along the supply chain, including seed contracts, equipment rental,
and sales, and who they source germplasm from for breeding and
exchanging seed. Surveys were hosted on the Qualtrics survey platform
and distributed over email. Each potential respondent was sent an
initial email invitation with three reminders, spaced out every two to
three weeks (Dillman et al.~2014).

Out of those we surveyed, we have responses from 94 seed producers, 49
companies, 60 academic and governmental researchers, and 44
organizations for a combined response rate of 33\% (247/756). The
response rates for each actor-group are shared in Table 3.1, and a more
extensive description of our sample\textquotesingle s representativeness in Appendix
C-1.
\begin{table}

\caption{\label{tab:unnamed-chunk-23}Organic seed innovation system population, sample, and survey response rate by survey group}
\centering
\begin{tabular}[t]{lrrr}
\toprule
Survey.group & N & n & Response.rate....\\
\midrule
Researcher & 117 & 60 & 51\\
Company & 130 & 49 & 38\\
Organization & 93 & 44 & 47\\
Producer & 416 & 94 & 23\\
Total & 756 & 247 & 33\\
\bottomrule
\end{tabular}
\end{table}
\hypertarget{challenges-and-expertise-as-system-resources}{%
\subsection{3.2 Challenges and expertise as system resources}\label{challenges-and-expertise-as-system-resources}}

We measure the presence of system resources using survey questions about
challenges and expertise among actors in the organic seed innovation
system. Surveys for seed producers and companies (including various
supply chain actors like breeders, processors and retailers) asked the
question: ``How much have the following been a challenge to you in your
work organic seed?'', followed by a list of 28 production and business
topics identified by experts in the Organic Seed Alliance. Surveys for
seed researchers and organizations were provided the same list of
topics, but instead asked: ``To what extent do you have expertise/work
with the following?''.

Of the 28 topics presented to respondents, we select 18 that are
relevant to knowledge and valuation resources and represent a unique
issue (full list is available in Appendix C-2). Examples of
knowledge-related topics include controlling weeds, managing climatic
effects, and irrigation. Examples of valuation-related topics include
developing infrastructure, sourcing appropriate equipment, and accessing
land. Respondents scored their perceived challenges or expertise on a
scale of 1 to 4. For challenges, the scale represented: 1 -- no
challenge, 2 -- somewhat of a challenge, 3 -- moderate challenge and 4
-- serious challenge. For expertise, the scale represented: 1 -- no
expertise/not my work, 2 -- low expertise/rarely work on, 3 -- moderate
expertise/sometimes work on, and 4 -- strong expertise/often work on.
Based on these survey questions we are able to compare the challenges
and expertise in the organic seed network and describe the dynamic of
knowledge and valuation resources at the systems level. We also
supplement these descriptive statistics with open-ended comments from
the surveys.

\hypertarget{network-construction-1}{%
\subsection{3.3 Network construction}\label{network-construction-1}}

We constructed the innovation system network using the survey questions
related to actors\textquotesingle{} connections. These connections were elicited using
the \textquotesingle hybrid\textquotesingle{} network question approach (Henry et al.~2012). Using this
approach, respondents were asked to list up to five to ten people or
organizations (number varied with question) that they look to, work
with, or source from across several seed-related contexts. For each name
they wrote, respondents had the option to select the types of
relationships. We match the types of relationships to the functional and
resource-based frameworks combined by Binz et al.~(2016). The knowledge
resource subsystem includes connections related to information
acquisition and exchange, research partnerships, stakeholder involvement
in projects, and academic collaborations. The valuation resource
subsystem combines the resource mobilization and market formation
functions, which includes seed acquisition, seed exchange, research
funding, licensing agreements, contracts, renting, buying, and/or
selling to/from. These ties are defined in Table 3.2.
\begin{table}

\caption{\label{tab:unnamed-chunk-24}Operationalization of innovation resource subsystems based on relationship types}
\centering
\begin{tabular}[t]{lll}
\toprule
Functional.subsystems & Resource.subsystems & Types.of.relationships\\
\midrule
Knowledge & Knowledge creation & Information acquisition and exchange, Research partnerships, stakeholder project involvement, and academic collaborations\\
Valuation & Resource mobilization & Seed acquisition, seed exchange, research funding\\
Valuation & Market formation & Licensing agreements, contracts, rent, buy, and/or sell to/from\\
\bottomrule
\end{tabular}
\end{table}
This survey-based method of network creation has benefits over methods
that derive relationships based on project or organization
co-participation, as is common in other innovation network studies
(Hermans 2013a; van alphen 2010, Binz et al.~2014). First, this method
does not rely on the assumption that actors who attend the same forum or
are members of the same organization are necessarily in collaboration,
as is the case with networks derived from co-participation (Berardo and
Scholz). Instead, this approach allows connections to be defined based
on actors\textquotesingle{} own perceptions of a relationship. Second, there are
challenges to analyzing two-mode data (i.e.~actor-to-project) with
traditional network methods (Jasny), which we bypass by collecting
one-mode data (i.e.~actor-to-actor). However, there are limits to our
approach. First, networks defined by survey responses will be
incomplete, given the commonly low survey response rates (). Second,
survey network collection is cross-sectional and so we cannot speak to
how the network evolves without repeating this data collection process.

The innovation system network includes a total of 1208 ties between 645
uniquely identifiable nodes. Based on data from the survey and
supplementary seed stakeholder databases, we added actor-type and
geographic attributes to each stakeholder in our network. We label the
actors by their role based on the types outlined by the multi-actor
perspective (Avelino and Wittmeyer). The initial groupings based on
these typologies are: state, firm, third sector (non-governmental
organizations), and research. We then divide the actors in our research
category into government agency actors and public university actors
because we argue that university researchers have their own
institutional logic (Leydesdorff). We also divide the \textquotesingle firm\textquotesingle{} category
into seed producers and companies. Although they are both actors under
the private institutional logic, we argue that producers (and farmers
generally) are a blend between for-profit and community logics, as their
occupation is often linked to their identity, place, and a particular
value set (Thompson et al.~2015). As a result, our actor groups and
their institutional logics are as follows: Seed producers
(firm-community hybrid), seed processors and retails (firms),
governmental agencies (state) and universities (state-community hybrid),
and non-governmental organizations (third sector). We also include the
geographic location and spatial scale in which a particular actor is
based (for more information see: Chapter 2).

\hypertarget{analytical-approach-ergms}{%
\subsection{3.4 Analytical approach: ERGMs}\label{analytical-approach-ergms}}

To test how structural motifs and actor composition vary across seed
innovation resource subsystems, we employ two Exponential Random Graph
Models (ERGMs) (Lusher et al.~2012). ERGMs are an inferential network
analysis method that estimate the likelihood that a network tie will
form based on endogenous and exogenous predictors (Robins et al.~2012).
Endogenous predictors are the structural features of the network itself,
such as the number of connections and nodes, open and closed triangle
formations, among many others. Exogenous predictors are features of the
actors in the network -- referred to in network terminology as \textquotesingle nodes\textquotesingle{}
-- which in this analysis include the actor types and geographic region
and scale. An ERGM estimates the likelihood that these features
influence tie formation in the observed network by comparing it to
thousands of simulated graphs that have the same structural parameters
as those specified in the model (Prell 2012).

The ERGMs we fit in this paper include five endogenous structural terms
and six exogenous node attributes (Table 3.3). Of focus for our first
hypothesis are two of the endogenous structural terms. These include a
structural parameter for a node\textquotesingle s degree (specified with geometric
weighting, \textquotesingle gwdegree\textquotesingle), which represents the likelihood of an actor
forming a tie given the number of ties it already has. In other words,
the likelihood of an \emph{additional} tie forming (Levy and Lubell 2018).
Based on our first hypothesis, we expect a positive value of this
parameter, which indicates a tendency away from centralization and more
open triangles (Levy 2016)\footnote{Note that this parameter is often misinterpreted in studies that
  use ERGMs and we encourage readers with further interest to explore
  more in the following resources:
  \href{https://figshare.com/articles/poster/Interpretation_of_GW-Degree_Estimates_in_ERGMs/3465020}{{[}https://figshare.com/articles/poster/Interpretation\_of\_GW-Degree\_Estimates\_in\_ERGMs/3465020{]}}}. Second, we include a structural
parameter for \textquotesingle triadic closure\textquotesingle{} (specified as geometrically weighted
edgewise shared partners, \textquotesingle gwesp\textquotesingle). This parameter represents the
likelihood of a connection to close a triangle, meaning to connect two
actors who share a connection to a third actor (Levy 2016; Ready and
Power 2020). We expect a positive value of this parameter, which
indicates a tendency of the network to have more closed triangles.

Of focus for our second hypothesis is the actor composition, an
exogenous node attribute. Each node in the network is associated with an
actor-type, and the relative activity of each actor type (propensity for
forming a tie) is compared against a baseline. In our analysis we set
seed producers as the comparison group, against which a parameter value
is estimated for each actor type representing their likelihood of
forming a tie. These estimations help us better understand what kinds of
groups are more or less active across the networks. Based on our second
hypothesis, we expect positive parameter values for non-profit actors in
the knowledge resource subsystem and a positive parameter value for
company actors in the valuation resource subsystem.

The remaining eight terms in the model are controls. The geographic
controls, regional homophily and geographic scale, account for the
within-region affinity that is prominent in the organic seed network and
the relative prominence of national and international actors (Chapter
2). The structural and design controls account for the influence of the
survey sampling method on the shape of the network, as well as how
connections in one network influence the connections in another via
multi-functional spillover (e.g.~horizontal coupling, see Chapter 2).

NEED TO PUT IN TABLE3

We estimate an ERGM for each of the two resource subsystems in order to
identify the variables that most influence network formation. We use the
`statnet` (Krivitsky et al.~2022) software package. All analysis was
conducted using R statistical software and code is available AT FINAL
LINK. Model building and selection is available in Appendix C-3.

\hypertarget{results-2}{%
\section{4. Results}\label{results-2}}

\hypertarget{organic-seed-system-resource-deficits}{%
\subsection{4.1 Organic seed system resource deficits}\label{organic-seed-system-resource-deficits}}

The organic seed system has been developing now for over twenty years
(Chapter 1, Hubbard et al.~2022), but it still faces several challenges
as it stabilizes. In this section we summarize the state of the
system\textquotesingle s knowledge and valuation resources based on challenges and
expertise reported by seed stakeholders.

The greatest challenges for producers and companies in the organic seed
system relate to knowledge deficits on the technical (generally
agronomic) strategies for organic seed production. Controlling weeds,
estimating and achieving adequate seed yields, and managing climate
effects are among the most serious challenges facing seed producers and
firms along the value chain (Figure 3.1A). Survey respondents provided
open-ended comments to reinforce these points, for example, ``Weed
control is a huge challenge and effect {[}sic{]} seed quality and yield.''
; ``Constantly changing climate; Rodents and other pests for roots and
legumes; Cabbage family pest Harlequin beetle''. Challenges related to
valuation resources are slightly less serious, though actors do face
issues such as developing infrastructure, accessing labor, production
costs, and sourcing appropriate equipment (Figure 3.1B). One respondent
noted their need for experienced labor by commenting that ``I would and
could expand the seed operation with more folks on board but not alone
as it is now.''

Between the two types of resources, the average rating of
knowledge-based challenges among respondents is 2.5 (between low and
moderate challenge), compared to the average rating of 2.2 for
valuation-based challenges. Though the difference is small, a comparison
of means using a Welch Two Sample t-test between the two groupings finds
a significant difference (p \textless{} 0.001), suggesting that there is a
significantly larger resource deficit for knowledge than valuation
resources.\{width=``6.5in''
height=``3.6527777777777777in''\}
\begin{figure}

{\centering \includegraphics[width=1\linewidth]{../osisn_processes/figures/figure3} 

}

\caption{Ratings of challenges (purple) and expertise (blue) for topics related to A. Knowledge resources and B. Valuation resources. Points represent the mean score from respondents 1-4 ratings (challenge n=144, expertise n=104) and gray bars represent the standard deviation from the mean.}\label{fig:unnamed-chunk-25}
\end{figure}
Though the challenges experienced by stakeholders point to moderate
resource deficits in the organic seed system, there are also expertise
in the network that address these same topics. Expertise is particularly
high on knowledge-related topics, such as identifying key traits,
controlling weeds, soil fertility, and disease pressure. There is also
low to moderate expertise available on valuation topics like developing
infrastructure, finding high quality seed stock, and accessing capital.

Between the two types of resources, the average rating of agronomic
knowledge-based expertise among respondents is 2.9 (moderate expertise),
compared to the average rating of 2.5 for valuation-based challenges.
Again, though the difference is small, a comparison of means between the
two groupings finds a significant difference (p \textless{} 0.001), suggesting
that there is a significantly higher average expertise for knowledge
than valuation topics.

Comparing challenges and expertise to assess system resources, we see
that the strength of expertise by researchers and organizations is
higher than the magnitude of challenges perceived by seed producers and
companies. On average, expertise related to knowledge resources score
0.4 points higher than challenges related to knowledge resources and
only one topic -- managing climatic effects -- has lower expertise than
challenges. Similarly, expertise related to valuation resources score
0.3 points higher than challenges related to valuation resources and
only one topic -- sourcing seed cleaning equipment -- has lower
expertise than challenges. Overall, the summaries suggest that the
organic seed system has the resources it needs to address its
challenges.

What we cannot say using these descriptive summaries, however, is how
well the supply and demand across these resources align in the network.
Some respondents report satisfaction with the resources available to
them: ``A few good books and a few field days under our belt and you can
do a lot of seed saving.'' In other cases, however, the quality and
relevance of the expertise in the system may be perceived as inadequate
by the stakeholders facing the challenges. For example, despite there
being considerable expertise in controlling disease pressure among
survey respondents, one seed producer remarks: \emph{``MORE OPEN-SOURCE info
on Pertinent seed-born diseases!! Easily searchable interface giving a
list of possible diagnosis and possible protocols. {[}Author\textquotesingle s{]} book
comes up fairly short on this. {[}Organization{]} has almost nothing for
one of the most critical attributes for quality seed production.''}
Furthermore, there are some producers that feel that existing resources
are not relevant for their production context. For example, respondents
note: \emph{``Hard to find people with sufficient knowledge and familiarity
with the crops I grow.''} and \emph{``I feel like the {[}Southeast{]} has its
own set of growing issues which are not always researched. They
especially complicate seed growing.''}

Based on the data we have on challenges and expertise across resources
in the organic seed system, we propose that both the knowledge and
valuation subsystems have resources that are existent but distributed.
Situating Hypothesis 1 in this case context, we expect both subsystems
to be built around closed triangles of the partner mode in accordance
with resource availability. Between the two subsystems, however, we
observe that the availability of valuation resources (expertise) is
lower than knowledge resources. As a result, we expect the valuation
subsystem to have fewer closed triangles relative to the knowledge
subsystem, suggesting a tendency towards an intermediary mode to help
address resource deficits.

\hypertarget{network-descriptives}{%
\subsection{4.2 Network descriptives}\label{network-descriptives}}

The organic seed innovation system identified through our survey methods
includes 645 stakeholders forming 1908 connections to one another.
Across the two resource subsystems, the knowledge subsystem is larger
compared to the valuation subsystem, where 522 actors account for 1106
ties compared to 418 actors accounting for 802 ties, summarized in Table
3.4. On average, an actor in the whole innovation system network and its
two subsystems will have 5.92, 4.24, and 3.84 connections, respectively.
The transitivity of each network, which summarizes the clustering of
nodes into triangles, is low and generally decreases with network size.
Last, the degree centralization of each network, which summarizes how
much the connections in the graph are organized around central/focal
points, is highest in the knowledge subnetwork and lowest in the
valuation subnetwork.
\begin{table}

\caption{\label{tab:unnamed-chunk-26}Descriptive summaries of the innovation system network and resource subsystems}
\centering
\begin{tabular}[t]{llll}
\toprule
X. & Innovation.system & Knowledge & Valuation\\
\midrule
Actors & 645 & 522 & 418\\
Connections & 1908 & 1106 & 802\\
Avg. \# connections & 5.92 & 4.24 & 3.84\\
Density & 0.006 & 0.006 & 0.007\\
Transitivity & 0.09 & 0.09 & 0.06\\
\addlinespace
Centralization & 0.1 & 0.12 & 0.08\\
Producers & 125 (19\%) & 101 (19\%) & 92 (22\%)\\
Company & 208 (32\%) & 146 (28\%) & 177 (42\%)\\
Organization & 190 (29\%) & 171 (33\%) & 86 (21\%)\\
University \&
extension & 73 (11\%) & 73 (14\%) & 33 (8\%)\\
\addlinespace
Government & 49 (8\%) & 31 (6\%) & 30 (7\%)\\
\bottomrule
\end{tabular}
\end{table}
Across the different actor types, companies and organizations each
account for nearly a third of the whole innovation system network,
producers a fifth, and university extension and government actors
together account for the remaining fifth. Representations vary across
the resource subsystems, however. Relative to the whole system,
for-profit actors like producers and companies account for a higher
percentage in the valuation subnetwork, and non-profit actors like
organizations and university researchers represent a greater fraction of
the nodes in the knowledge subnetwork.

The seed innovation system network and its two resource subsystems are
visualized in Figure 3.4. Across nearly all \textquotesingle slices\textquotesingle{} of the system,
the networks are mostly connected as one component, meaning nearly every
node can connect to other nodes by different \textquotesingle paths\textquotesingle. Across the
subsystems there is decently strong correlation (0.41), suggesting that
most actors use connections for multiple purposes that support both
knowledge and valuation resources.

These structural summaries provide an overview of the organic seed
network, but we cannot use them to make inferences about what drives
these structures, nor generalize them to speak more broadly about the
tendencies of structural formation across different relationships. In
the next section, we turn to the results of our structurally explicit
network analysis to elaborate on the relationship between structure and
resource needs in innovation systems.

\{width=``6.5in'' height=``2.0in''\}
\begin{figure}

{\centering \includegraphics[width=1\linewidth]{../osisn_processes/figures/figure4} 

}

\caption{The resource subsystems of the organic seed innovation system. A) Knowledge subsystem, B) Valuation subsystem and C) Whole innovation system. Circles in the graph represent actors/organizations and lines represent connections based on the different relationships within each subsystem. Nodes are sized by their relative popularity in the figure (the larger, the greater number of ties they have). Nodes are colored based on their role (outlined in the legend), with the color gradient ranging from more private (darker) to more public (lighter).  }\label{fig:unnamed-chunk-27}
\end{figure}
\hypertarget{network-inferences-with-ergms}{%
\subsection{4.3 Network inferences with ERGMs}\label{network-inferences-with-ergms}}

We address our hypotheses about the structure and composition of the
organic seed innovation subsystems using results from the Exponential
Random Graph Models. An ERGM estimates the likelihood of network tie
formation, providing statistically supported explanations of why network
structures form. Model results presented in Figure 3.5 show coefficient
estimates and their confidence intervals as log-odds. When reading
log-odds, a basic intuition is that values less than zero represent
lower probabilities of forming a tie and values greater than zero
represent higher probabilities of forming a tie. To interpret specific
statistics throughout the text, we transform the log-odds into odds and
evaluate the difference between the coefficient value and one in order
to describe the likelihood of a tie as a percentage (Scott and Thomas
2015; Ulibarri and Scott 2017). A table of model results is available in
Appendix C-4.

\hypertarget{resource-subsystem-structure}{%
\subsubsection{4.3.1 Resource subsystem structure}\label{resource-subsystem-structure}}

In this section we present results related to our first hypothesis:
\emph{Closed triangles/partner mode network structures will be more prevalent
in innovation resource subsystems that have existing, decentralized
resources} As described in Section 4.1, the organic seed innovation
system has existing, distributed knowledge and valuation resources, and
so we expect that both subnetworks will show closed triangles measured
by a positive triadic closure parameter, rather than open triangles
measured by a negative de-centralization parameter. Between the two
subsystems, however, we expect the valuation subsystem to have fewer
closed triangle structures, as it tends toward the more centralized,
intermediary mode to address its deficits.

The variables relevant to this hypothesis are anti-centralization and
triadic closure (Figure 3.5). For the anti-centralization coefficients,
we observe positive, significant, and large estimates for both resource
subsystems (knowledge: 5.31, SE: 0.52 ; valuation: 4.34, SE: 0.47).
Similarly, for the transitivity parameter we observe positive and
significant coefficient estimates for both resource subsystems
(knowledge: 0.77, SE: 0.06 ; valuation: 0.4, SE: 0.06).

On the whole, these structural results align with what we expected in
our first hypothesis. These resource subsystems are relying on the
partner mode of system resource formation through closed triangle
structures to connect actors to available resources. For example, based
on the transitivity term, the odds of a tie between two actors with a
shared partner are 116\% greater than the odds of a tie between two
actors with no shared partners in the knowledge subnetwork. And based on
the anti-centralization term, the odds of an actor forming a tie with
another actor that already has two connections in the valuation
subsystem are 31\% lower than the odds of that same actor forming a tie
with another actor that only has one existing connection. Put more
simply, rather than forming open triangle ties with a few central,
coordinating actors, actors are building more diffuse relationships for
cooperative, partnered relationships.

\{width=``6.5in'' height=``4.333333333333333in''\}
\begin{figure}

{\centering \includegraphics[width=1\linewidth]{../osisn_processes/figures/figure5} 

}

\caption{Coefficient plot for the ERGMs of the knowledge (dark purple, closed circles) and valuation (blue, open circles) subsystems. Each point represents the value of the coefficient estimate (reported as log-odds) and bars represent the confidence intervals surrounding each estimate. Values with confidence intervals fully on either size of zero are considered significant.}\label{fig:unnamed-chunk-28}
\end{figure}
When comparing the structures of the knowledge and valuation subsystems
to one another, we observe some variation that lends further support for
the resource-based theory. The models estimate that the knowledge
subsystem, which has more expertise (resources) than the valuation
subsystem, has more closed triangles and lower centralization (higher
anti-centralization) than the valuation subsystem. While the odds of a
tie forming between two actors with a shared partner are 116\% greater
than the odds of a tie between two actors with no shared partners in the
knowledge subsystem, these odds decrease to 50\% in the valuation
subsystem. Likewise, actors in the knowledge subsystem are 37\% less
likely to form a tie with a popular, coordinating actor, in contrast
from the 31\% lower likelihood in the valuation subnetwork.

\hypertarget{resource-subsystem-composition}{%
\subsubsection{4.3.2 Resource subsystem composition}\label{resource-subsystem-composition}}

In this section we present results related to our second set of
hypotheses, that non-profit stakeholders are more active in knowledge
subsystems (H2a), while for-profit stakeholders are more active in
valuation subsystems (H2b). The \textquotesingle actor-type\textquotesingle{} variables in Figure 3.5
represent the likelihood of each actor-type forming a tie compared to
our baseline group of seed producers. For the knowledge subsystem,
university and organization actors are significantly more active than
the baseline, with coefficient estimates of 1.13 (SE 0.13) and 0.92 (SE
0.13) respectively. Converting these estimates into odds, the odds of
these actors\textquotesingle{} forming a tie is 210\% (university) and 153\%
(organization) greater than the odds of seed producers forming a tie in
the knowledge subsystem. Neither government nor company actors, however,
are particularly active in knowledge subsystems. These results partially
support H2a, given the strong activity by university and organizational
actors, but the average activity levels of government actors did not
align with our expectations.

For the valuation subsystem, both companies and government actors are
significantly more active than the baseline, with coefficient estimates
of 0.6 (SE 0.12) and 1.45 (SE 0.17) respectively. In other words, the
odds of these actors forming a tie is 88\% (companies) and 327\%
(governments) greater than the odds of seed producers. On the flip side,
organizations are significantly less likely to participate in these
networks, with a coefficient estimate of -0.47 (SE 0.16). These results
partially support H2b, given the strong activity by companies, however
the prominent role of government was not as expected.

\hypertarget{discussion-1}{%
\section{5. Discussion}\label{discussion-1}}

\hypertarget{conditions-for-innovation-system-formation}{%
\subsection{5.1 Conditions for innovation system formation}\label{conditions-for-innovation-system-formation}}

Our results support the resource-based theory that the existence (or
deficit) of resources help us predict the kinds of network structures
that will form within an innovation system (Musiolik et al.~2020). We
find that when resources are existent but diffuse, the partner mode of
network building -- operationalized as closed triangles in social
network and network governance literature () -- is more likely to
structure the network. This is true across both the knowledge and
valuation resource subsystems related to organic seed, which represent
the multiple functionalities of network structure within an innovation
system.

Though this work is among the first to empirically test conditions
predicting network structure in the innovation systems literature (Binz,
Cholez), we draw on social network and network governance fields to
quantify and contextualize these results in related disciplines (). We
see strong theoretical similarities between the resource-based theory
and the theory of network governance, which identifies the need for
network-level resources as an important determining factor in the choice
of network structure (Provan and Kenis, Lubell, Rudnick). When there are
network-level resources required, such as building trust and reputation,
or coordinating actors to create something new outside of the existing
expertise, the intermediary mode of open triangles with a central
coordinator is more useful. On the other hand, in the organic seed
network where the task is about connecting existing resources and
creating complementary alliances, the partner-mode of closed triangles
is more useful.

Connecting innovation system studies to the social network and network
governance fields opens up new lines of questioning related to the
conditions of network formation. For example, there are likely other
system features that predict network structure, such as size, trust,
goal consensus, task complexity, level of uncertainty, and customization
(Provan and Kenis, Jones). Some of these have been theorized and
explored, such as the customization of products affecting structure
across spatial scales(Binz and Truffer, Chapter 2), and how trust and
network size change over time (Hermans et al.~2013). Still, considerably
more research is needed to build out and test these conditions in the
context of innovation systems. Further, the methods for testing these
conditions need to become more consistent across the field to support
comparable results. We propose quantitative and inferential network
analysis, including methods like conditional uniform graph tests,
exponential random graph models, and autologistic actor attribute models
(Robins), as an approach for rigorously testing network building theory
(Scott and Ulibarri).

\hypertarget{actor-composition-aligns-with-logic-in-the-niche-phase}{%
\subsection{5.2 Actor composition aligns with logic in the niche phase}\label{actor-composition-aligns-with-logic-in-the-niche-phase}}

While resource-based needs shape the overall network structure of the
organic seed resource subsystems, we find that actors\textquotesingle{} institutional
logics also shape relationship-building activity. Our results largely
support the hypotheses defined by the multi-actor perspective (). We
find that non-profit actors like organizations and universities are
significantly more active in the knowledge resource subsystem, while
for-profit companies are more active in the valuation resource
subsystem. These results fit with the understanding that there are
pre-competitive (knowledge creation) and competitive (knowledge
exploitation and value creation) divisions among public/non-profit and
private/for-profit actors (King/Auti). These results help us better
understand what it means to be a \textquotesingle system builder\textquotesingle{} (Musiolik). In
short, system builder profiles likely change based on the kinds of
resources that are being generated (Leysdorff). Activity is likely to be
higher among actors whose incentive structures align with the system
resources being created in the network.

Alignment between institutional logic and resource subsystem met our
expectations for all of the actor types except government. Rather than
fitting cleanly into the role of public actor with a pre-competitive
logic, governments played a blended role. Governmental actors and
organizations were considerably more active than the baseline in the
valuation subsystem, but not the knowledge subsystem. Though these
results defy the more traditional logics of government (MAP/Leysdorff),
these results make sense in the context of a niche innovation system
(Raven). Because niches are often not yet competitive in their own
right, the government often plays a nurturing role to support
traditionally private functions (Raven). For instance, the breeding
material shared by the government\textquotesingle s national and regional repositories
are a central contributor to the supply chain, providing important
genetic inputs for R\&D and plant variety development early in the
innovation process.

These findings advance our understanding of actor agency by mapping
system-building motivations to actors\textquotesingle{} institutional logics. While
actor composition has been of interest to innovation systems, rarely has
this work been able to generalize findings about certain types of actors
leading particular functions (\textbf{Lamers et al., 2017}). Understanding
not only \emph{what} structures are forming, but also \emph{who} is forming those
structures can help policymakers identify what kinds of stakeholders to
support for different types of innovation system development.

\hypertarget{limitations-1}{%
\subsection{5.3 Limitations}\label{limitations-1}}

The theoretical and methodological contributions of this paper are
certainly not without limitations. First, we took the partner and
intermediary mode concepts from resource-based theory and translated
them into the social network motifs. In this there is room for
interpretive error. We relied on the intermediary mode description from
Musiolik et al.~(2020) where \textquotesingle a system builder coordinates activities
in a network of collaborators to develop new, intermediate
organizational structures\textquotesingle{} (p.~10). We believe this aligns best with
Provan and Kenis\textquotesingle s description of \textquotesingle lead\textquotesingle{} and \textquotesingle network
administrative\textquotesingle{} organizations for governance, which map onto open
triangle structures. However, there is room for interpretation.
Specifically, the intermediary mode is also described as \textquotesingle risk
sharing\textquotesingle{} (Table 2, p.~9) where \textquotesingle system resource build-up is
demanding\textquotesingle{} (Table 3, p.~11). Other theoretical perspectives, such as
the risk hypothesis in policy network literature (Berardo and Scholz
2010), associate these features with closed triangle structures.

Furthermore, there is no clear definition of resource deficit in the
resource-based literature (), and so our operationalization of this
concept in the organic seed system was difficult. As mentioned in
Section 4.1, we are limited in assessing the quality of the expertise,
and in understanding how actors with different challenges and expertise
align in the network. So while we conclude that resources do exist
across the system, we recognize that these resources may not be evident,
accessible, or relevant to the actors who need it. Future studies should
do more to clarify how to define resource availability and deficits in
order to strengthen testing of the resource-based theory.

\hypertarget{conclusion-1}{%
\section{6. Conclusion}\label{conclusion-1}}

This paper seeks to understand the conditions predicting innovation
system network formation and actor composition. We test existing
theories about resource-based innovation networks (Musiolik) and actor
logics (Avelino) by connecting to social network theory and methods
(Provan and Kenis, Robins). We use the case of the organic seed
innovation system, which we divide into two resource subsystems
representing knowledge and valuation functions, both of which have
existing but dispersed resources. Exponential Random Graph Modeling
helps us statistically test the likelihood of tie formation to describe
different network building processes and relative actor involvement.
These models show support for the idea that resource constellations are
important conditions shaping network structure. Additionally, we find
that different actors\textquotesingle{} activity in resource subsystems depends on their
incentive structure -- public actors generally are engaged in knowledge
resource building as that function better aligns with the
pre-competitive logic of public institutions. On the flip side, private
actors are generally more engaged in building valuation resource
subsystems.

Understanding the processes by which innovation systems are formed is
the first step towards improving system performance (Musiolik et al.
2012). This is especially important for informing policymakers about
what types of network structuring modes are most relevant given a
system\textquotesingle s state of resources, and what types of actors are more or less
likely to be active participants and leaders in these networks. In
organic seed, for example, rather than needing new, intermediary
structures, the system needs support for identifying actors with
complementary resources and support for building cooperative
relationships. Additionally, at this stage in the niche\textquotesingle s development,
policies should recognize the important role of government-backed
resources like germplasm, and continue to support the provision of these
public goods that support the niche supply chain (Hubbard 2023).
Altogether, these findings can help both researchers and policymakers
better understand the circumstances under which certain system building
strategies should be put into place.

\hypertarget{conclusion-2}{%
\chapter*{Conclusion}\label{conclusion-2}}
\addcontentsline{toc}{chapter}{Conclusion}

If we don't want Conclusion to have a chapter number next to it, we can add the \texttt{\{-\}} attribute.

\textbf{More info}

And here's some other random info: the first paragraph after a chapter title or section head \emph{shouldn't be} indented, because indents are to tell the reader that you're starting a new paragraph. Since that's obvious after a chapter or section title, proper typesetting doesn't add an indent there.

\appendix

\hypertarget{appendix-a}{%
\chapter{Appendix A}\label{appendix-a}}

\hypertarget{model-selection-and-fit}{%
\section{1. Model selection and fit}\label{model-selection-and-fit}}
\begin{table}

\caption{\label{tab:unnamed-chunk-29}Model comparison results}
\centering
\resizebox{\linewidth}{!}{
\begin{tabular}[t]{lllllllllrr}
\toprule
\multicolumn{1}{c}{ } & \multicolumn{2}{c}{Model 1} & \multicolumn{2}{c}{Model 2} & \multicolumn{2}{c}{Model 3} & \multicolumn{2}{c}{Model 4} & \multicolumn{2}{c}{Model 5} \\
\cmidrule(l{3pt}r{3pt}){2-3} \cmidrule(l{3pt}r{3pt}){4-5} \cmidrule(l{3pt}r{3pt}){6-7} \cmidrule(l{3pt}r{3pt}){8-9} \cmidrule(l{3pt}r{3pt}){10-11}
Term & Mean & SE & Mean & SE & Mean & SE & Mean & SE & Mean & SE\\
\midrule
Barrier: Organic availability &  &  &  &  & 0.242 & 0.001 & 0.173 & 0.001 & 0.171 & 0.001\\
Barrier: Undesirable traits &  &  &  &  & 0.105 & 0.001 & 0.124 & 0.001 & 0.121 & 0.001\\
Barrier: Buyer requirements &  &  &  &  & 0.092 & 0.001 & 0.126 & 0.001 & 0.121 & 0.001\\
Barrier: Insufficient quantity &  &  &  &  & 0.028 & 0.001 & 0.018 & 0.001 & 0.016 & 0.001\\
Year: 2015 (vs. 2010) & -0.371 & 0.002 &  &  &  &  & -0.19 & 0.002 & -0.175 & 0.002\\
\addlinespace
Year: 2020 (vs. 2010) & -0.413 & 0.002 &  &  &  &  & -0.309 & 0.002 & -0.316 & 0.002\\
Values organic seed &  &  & -0.27 & 0.001 & -0.202 & 0.001 & -0.202 & 0.001 & -0.200 & 0.001\\
Farm size (acres) &  &  & 0.072 & 0.001 & 0.029 & 0.001 & 0.106 & 0.001 & 0.118 & 0.001\\
Crop diversity &  &  & -0.035 & 0.001 & -0.064 & 0.001 & -0.071 & 0.001 & -0.068 & 0.001\\
Seed saved or traded (\%) &  &  &  &  &  &  &  &  & -0.095 & 0.001\\
\addlinespace
Certifiers request organic seed use &  &  &  &  &  &  &  &  & 0.038 & 0.001\\
Forage crops (vs. Field) & 0.82 & 0.002 &  &  &  &  & 0.932 & 0.002 & 0.903 & 0.002\\
Vegetable crops (vs. Field) & 0.186 & 0.002 &  &  &  &  & 0.286 & 0.002 & 0.248 & 0.002\\
Western region (vs. North Central) & 0.439 & 0.002 &  &  &  &  & 0.3 & 0.002 & 0.338 & 0.002\\
Northeastern region (vs. North Central) & 0.325 & 0.002 &  &  &  &  & 0.25 & 0.002 & 0.249 & 0.002\\
\addlinespace
Southern region (vs. North Central) & 0.252 & 0.003 &  &  &  &  & 0.116 & 0.003 & 0.122 & 0.003\\
(Intercept) & -0.595 & 0.002 & -0.455 & 0.001 & -0.466 & 0.001 & -0.702 & 0.002 & -0.701 & 0.002\\
\bottomrule
\end{tabular}}
\end{table}
\begin{table}

\caption{\label{tab:unnamed-chunk-30}Goodness of fit across models}
\centering
\begin{tabular}[t]{lrrr}
\toprule
Model & WAIC & Maximum Likelihood & CPO\\
\midrule
Model 1 & -4310 & 2114 & 1.23\\
Model 2 & -4247 & 2100 & 1.21\\
Model 3 & -4354 & 2130 & 1.24\\
Model 4 & -4478 & 2157 & 1.28\\
Model 5 & -4484 & 2148 & 1.28\\
\bottomrule
\end{tabular}
\end{table}
\hypertarget{full-model-results}{%
\section{2. Full model results}\label{full-model-results}}
\begin{table}

\caption{\label{tab:unnamed-chunk-31}Full model results}
\centering
\resizebox{\linewidth}{!}{
\begin{tabular}[t]{lrr}
\toprule
Term & Mean estimate & Standard error\\
\midrule
(Intercept) & -0.663 & 0.002\\
Southern region (vs. North Central) & 0.124 & 0.003\\
Northeastern region (vs. North Central) & 0.236 & 0.002\\
Western region (vs. North Central) & 0.351 & 0.002\\
Vegetable crops (vs. Field) & 0.214 & 0.002\\
\addlinespace
Forage crops (vs. Field) & 0.848 & 0.002\\
Certifiers request organic seed use & 0.043 & 0.001\\
Seed saved or traded (\%) & -0.096 & 0.001\\
Crop diversity & -0.065 & 0.001\\
Farm size (acres) & 0.114 & 0.001\\
\addlinespace
Values organic seed & -0.200 & 0.001\\
Year: 2020 (vs. 2010) & -0.334 & 0.002\\
Year: 2015 (vs. 2010) & -0.171 & 0.002\\
Barrier: Insufficient quantity & 0.009 & 0.001\\
Barrier: Buyer requirements & 0.112 & 0.001\\
\addlinespace
Barrier: Undesirable traits & 0.127 & 0.001\\
Barrier: Organic availability & 0.140 & 0.001\\
\bottomrule
\end{tabular}}
\end{table}
\hypertarget{predicted-conventional-seed-use-for-field-and-forage-crop-growers}{%
\section{3. Predicted conventional seed use for field and forage crop growers}\label{predicted-conventional-seed-use-for-field-and-forage-crop-growers}}
\begin{figure}

{\centering \includegraphics[width=1\linewidth]{../organicseed_adoption/figures/figurea1} 

}

\caption{Predicted mean estimates for field crop growers of two farmer profiles.}\label{fig:unnamed-chunk-32}
\end{figure}
\begin{figure}

{\centering \includegraphics[width=1\linewidth]{../organicseed_adoption/figures/figurea2} 

}

\caption{Predicted mean estimates for forage crop growers of two farmer profiles.}\label{fig:unnamed-chunk-33}
\end{figure}
\hypertarget{appendix-b}{%
\chapter{Appendix B}\label{appendix-b}}

\hypertarget{selecting-spatial-terms}{%
\section{1. Selecting spatial terms}\label{selecting-spatial-terms}}

We further explore space's influence on tie formation by comparing the regional homophily model with two other spatial models.

First, we test the spatial relationship of actors based on geodesic distances between one actor and another, rather than defined regional boundaries. This allows us to better understand whether within-region homophily is merely a matter of convenience (e.g.~I get information from my neighbor because it is easier), rather than related to spatial embeddedness of the innovation. Actors for whom we don't have exact coordinates, we assign the mean geodesic distance. Our model shows a significant negative relationship between distances and likelihood of connection, suggesting that connections are not made only from convenience of distance. Instead, actor seek others who have relevance to the resources they need, but still in the regional context that they need it.
\begin{table}

\caption{\label{tab:unnamed-chunk-34}Full model results: Geodesic distance covariate}
\centering
\resizebox{\linewidth}{!}{
\begin{tabular}[t]{lrrrr}
\toprule
\multicolumn{1}{c}{ } & \multicolumn{2}{c}{Knowledge subsystem} & \multicolumn{2}{c}{Valuation subsystem} \\
\cmidrule(l{3pt}r{3pt}){2-3} \cmidrule(l{3pt}r{3pt}){4-5}
Term & Estimate & SE & Estimate & SE\\
\midrule
Spatial distance (log) & -0.081 & 0.027 & 0.062 & 0.045\\
Scale: National (vs. Regional) & -0.096 & 0.109 & 0.657 & 0.066\\
Other subsystem structure & 5.273 & 0.134 & 0.326 & 0.360\\
Anti-centralization & 4.997 & 0.497 & 3.682 & 0.406\\
Triadic closure & 0.964 & 0.062 & 0.802 & 0.058\\
\addlinespace
Edges & -4.689 & 0.387 & -6.091 & 0.660\\
Survey non-respondent & -1.170 & 0.142 & -1.104 & 0.107\\
N respondents per node & 0.280 & 0.026 & 0.121 & 0.027\\
N generic connections & -0.090 & 0.017 & 0.025 & 0.010\\
Fixed -Inf & -Inf & 0.000 & -Inf & 0.000\\
\bottomrule
\end{tabular}}
\end{table}
Second, we run the model using 12 Plant Hardiness Zones outlined by the USDA, rather than the four administrative regions. These PHZs help zoom in on more specific similarities across space regarding growing needs. We do not have PHZ for Canada, and so we look for the nearest USDA PHZ and assign it to them. Actors for whom we cannot assign a PHZ are given an `other' category. Similar to the four administrative regions, we see also homophily within the PHZs.
\begin{table}

\caption{\label{tab:unnamed-chunk-35}Full model results: Plant Hardiness Zones (PHZ)}
\centering
\resizebox{\linewidth}{!}{
\begin{tabular}[t]{lrrrr}
\toprule
\multicolumn{1}{c}{ } & \multicolumn{2}{c}{Knowledge subsystem} & \multicolumn{2}{c}{Valuation subsystem} \\
\cmidrule(l{3pt}r{3pt}){2-3} \cmidrule(l{3pt}r{3pt}){4-5}
Term & Estimate & SE & Estimate & SE\\
\midrule
PHZ homophily: Zone 3 & 2.819 & 1.226 & 2.518 & 0.895\\
PHZ homophily: Zone 4 & 1.729 & 0.250 & 1.141 & 0.255\\
PHZ homophily: Zone 5 & 1.459 & 0.171 & 0.703 & 0.197\\
PHZ homophily: Zone 6 & 0.921 & 0.180 & 0.886 & 0.158\\
PHZ homophily: Zone 7 & 1.093 & 0.270 & 1.445 & 0.246\\
\addlinespace
PHZ homophily: Zone 8 & 0.946 & 0.155 & 1.280 & 0.120\\
PHZ homophily: Zone 9 & 1.161 & 0.352 & 0.690 & 0.364\\
PHZ homophily: Zone 10 & 1.405 & 0.855 & 2.447 & 0.338\\
Scale: National (vs. Regional) & 0.039 & 0.094 & 0.696 & 0.066\\
Other subsystem structure & 5.170 & 0.188 & 0.390 & 0.406\\
\addlinespace
Anti-centralization & 5.041 & 0.465 & 3.742 & 0.433\\
Triadic closure & 0.921 & 0.065 & 0.757 & 0.062\\
Edges & -6.267 & 0.161 & -5.435 & 0.128\\
Survey non-respondent & -1.163 & 0.133 & -1.250 & 0.112\\
N respondents per node & 0.303 & 0.029 & 0.137 & 0.034\\
\addlinespace
N generic connections & -0.072 & 0.015 & 0.028 & 0.010\\
Fixed -Inf & -Inf & 0.000 & -Inf & 0.000\\
\bottomrule
\end{tabular}}
\end{table}
We compare the model fit of these three approaches and find best fit for the four-region model, presented in Appendix C-3 and Figure 2.3
\begin{table}

\caption{\label{tab:unnamed-chunk-36}Model comparison results}
\centering
\begin{tabular}[t]{lrrrr}
\toprule
\multicolumn{1}{c}{ } & \multicolumn{2}{c}{Knowledge subsystem} & \multicolumn{2}{c}{Valuation subsystem} \\
\cmidrule(l{3pt}r{3pt}){2-3} \cmidrule(l{3pt}r{3pt}){4-5}
Model & BIC & AIC & BIC & AIC\\
\midrule
Geodesic distance & 6179 & 6090 & 6590 & 6505\\
Plant Hardizess Zones & 6070 & 5913 & 6565 & 6415\\
SARE Regions & 5855 & 5727 & 6377 & 6255\\
\bottomrule
\end{tabular}
\end{table}
\hypertarget{defining-actors-spatial-scales}{%
\section{2. Defining actors' spatial scales}\label{defining-actors-spatial-scales}}

\hypertarget{full-model-results.}{%
\section{3. Full model results.}\label{full-model-results.}}
\begin{table}

\caption{\label{tab:unnamed-chunk-37}Full model results: SARE regions}
\centering
\resizebox{\linewidth}{!}{
\begin{tabular}[t]{lrrrr}
\toprule
\multicolumn{1}{c}{ } & \multicolumn{2}{c}{Knowledge subsystem} & \multicolumn{2}{c}{Valuation subsystem} \\
\cmidrule(l{3pt}r{3pt}){2-3} \cmidrule(l{3pt}r{3pt}){4-5}
Term & Estimate & SE & Estimate & SE\\
\midrule
Regional homophily: Northeast & 2.612 & 0.184 & 2.475 & 0.212\\
Regional homophily: South & 1.940 & 0.158 & 1.969 & 0.188\\
Regional homophily: North Central & 2.090 & 0.150 & 1.676 & 0.177\\
Regional homophily: West & 1.076 & 0.146 & 1.518 & 0.132\\
National homophily: US National & -0.938 & 0.399 & -0.916 & 0.272\\
\addlinespace
Scale: National (vs. Regional) & 0.735 & 0.122 & 1.389 & 0.092\\
Other subsystem structure & 5.135 & 0.137 & 0.357 & 0.378\\
Anti-centralization & 5.097 & 0.510 & 3.885 & 0.428\\
Triadic closure & 0.864 & 0.058 & 0.763 & 0.056\\
Edges & -6.587 & 0.170 & -5.989 & 0.153\\
\addlinespace
Survey non-respondent & -1.175 & 0.146 & -1.103 & 0.112\\
N respondents per node & 0.278 & 0.030 & 0.129 & 0.029\\
N generic connections & -0.078 & 0.017 & 0.029 & 0.009\\
Fixed -Inf & -Inf & 0.000 & -Inf & 0.000\\
\bottomrule
\end{tabular}}
\end{table}
\begin{figure}

{\centering \includegraphics[width=1\linewidth]{../osisn_spatial/figures/figureb3_kgof} 

}

\caption{Knowledge resource subsystem ERGM goodness of fit plots}\label{fig:unnamed-chunk-38}
\end{figure}
\begin{figure}

{\centering \includegraphics[width=1\linewidth]{../osisn_spatial/figures/figureb3_vgof} 

}

\caption{Valuation resource subsystem ERGM goodness of fit plots}\label{fig:unnamed-chunk-39}
\end{figure}
\hypertarget{appendix-c}{%
\chapter{Appendix C}\label{appendix-c}}

\hypertarget{sample-representativeness-across-us-regions-and-spatial-scales}{%
\section{1. Sample representativeness across US regions and spatial scales}\label{sample-representativeness-across-us-regions-and-spatial-scales}}
\begin{table}

\caption{\label{tab:unnamed-chunk-40}Sample representativeness across US regions and spatial scales}
\centering
\resizebox{\linewidth}{!}{
\begin{tabular}[t]{lrrrrrrrrrrrr}
\toprule
\multicolumn{1}{c}{ } & \multicolumn{3}{c}{Producer} & \multicolumn{3}{c}{Company} & \multicolumn{3}{c}{Researcher} & \multicolumn{3}{c}{Organization} \\
\cmidrule(l{3pt}r{3pt}){2-4} \cmidrule(l{3pt}r{3pt}){5-7} \cmidrule(l{3pt}r{3pt}){8-10} \cmidrule(l{3pt}r{3pt}){11-13}
Location & n & N & \% response & n & N & \% response & n & N & \% response & n & N & \% response\\
\midrule
Total & 94 & 416 & 23 & 49 & 130 & 38 & 60 & 117 & 51 & 44 & 93 & 47\\
West & 53 & 242 & 22 & 13 & 40 & 32 & 13 & 28 & 46 & 10 & 24 & 42\\
North Central & 13 & 74 & 18 & 7 & 17 & 41 & 16 & 34 & 47 & 12 & 18 & 67\\
Northeast & 9 & 39 & 23 & 7 & 10 & 70 & 8 & 14 & 57 & 3 & 12 & 25\\
South & 11 & 38 & 29 & 2 & 7 & 29 & 21 & 38 & 55 & 7 & 11 & 64\\
\addlinespace
USA & 0 & 4 & 0 & 17 & 45 & 38 & 1 & 1 & 100 & 5 & 14 & 36\\
Canada & 6 & 16 & 38 & 2 & 8 & 25 & 1 & 2 & 50 & 6 & 9 & 67\\
Other country & 2 & 2 & 100 & 0 & 0 & NA & 0 & 0 & NA & 0 & 0 & NA\\
International & 0 & 1 & 0 & 1 & 3 & 33 & 0 & 0 & NA & 1 & 5 & 20\\
\bottomrule
\end{tabular}}
\end{table}
\hypertarget{full-list-of-resource-related-topics}{%
\section{2. Full list of resource-related topics}\label{full-list-of-resource-related-topics}}

\textbf{Knowledge resource topics}
Finding high quality stock seed
Achieving adequate seed yields
Isolation distances
Soil fertility and crop nutrition
Irrigation and water use
Controlling weeds
Controlling insect pests
Controlling disease pressure
Managing climatic effects
Estimating yields -- redundancy with Achieving adequate seed yields
Adapting to climate change -- redundant with Managing climatic effects
Managing pollinator habitats -- too niche for many responses
Vernalization for biennial crops -- too niche for many responses
Overwintering for biennial crops -- too niche for many responses
In-field seed production costs -- irrelevant to knowledge or valuation resources
Harvest costs -- irrelevant to knowledge or valuation resources
Seed cleaning costs -- irrelevant to knowledge or valuation resources

\textbf{Valuation resource topics}
Accessing labor
Accessing land
Accessing capital
Farm business planning
Developing infrastructure
Finding/developing markets
Managing intellectual property rights
Sourcing seed harvest equipment
Sourcing seed cleaning equipment
Requirements of organic certification -- irrelevant to knowledge or valuation resources
Contamination from GE crops -- irrelevant to knowledge or valuation resources

\hypertarget{model-building-and-selection}{%
\section{3. Model building and selection}\label{model-building-and-selection}}

First, selecting decay parameters will full model. The models do not converge when running several all the possible iterations. So first I run a range of \texttt{gwdegree} decay values with a \texttt{gwesp} decay value = 0.5. Then based on the best-performing \texttt{gwdegree} decay value (alpha = 0.1), I run a range of \texttt{gwesp} decay values.
\begin{figure}

{\centering \includegraphics[width=1\linewidth]{../osisn_processes/figures/figurec1} 

}

\caption{BIC values across different ERGM geometrically weighted decay value specifications}\label{fig:unnamed-chunk-41}
\end{figure}
Based on decay values of 0.7 (knowledge) 0.9 (valuation), I compare different builds of the model.
\begin{table}

\caption{\label{tab:unnamed-chunk-42}ERGM comparison for knowledge resource subsystem}
\centering
\resizebox{\linewidth}{!}{
\begin{tabular}[t]{llllll}
\toprule
Coefficient & Model1 & Model2 & Model3 & Model4 & Model5\\
\midrule
Anti-centralization &  & 4.89***
(0.46) & 4.91***
(0.68) & NA
(NA) & 5.31***
(0.52)\\
Triadic closure &  & 1.27***
(0.04) & 0.8***
(0.07) & NA
(NA) & 0.77***
(0.06)\\
Actor-type: Government &  &  &  & NA
(NA) & -0.14
(0.28)\\
Actor-type: University \& extension &  &  &  & NA
(NA) & 1.13***
(0.13)\\
Actor-type: Organization &  &  &  & NA
(NA) & 0.93***
(0.13)\\
\addlinespace
Actor-type: Company &  &  &  & NA
(NA) & -0.13
(0.12)\\
Regional homophily: Northeast &  &  & 2.67***
(0.32) & NA
(NA) & 2.6***
(0.23)\\
Regional homophily: South &  &  & 2.52***
(0.35) & NA
(NA) & 1.96***
(0.16)\\
Regional homophily: North Central &  &  & 2.18***
(0.26) & NA
(NA) & 1.85***
(0.14)\\
Regional homophily: West &  &  & 1.3***
(0.26) & NA
(NA) & 1.44***
(0.15)\\
\addlinespace
National homophily: US National &  &  & -1.23**
(0.61) & NA
(NA) & -0.9
(0.48)\\
Scale: National \& International (vs. Regional) &  &  & 0.89***
(0.19) & NA
(NA) & 0.91***
(0.12)\\
Multi-functional tie &  &  & 5.44***
(0.19) &  & 5.77***
(0.17)\\
Survey non-respondent &  & -1.37***
(0.1) & -1.31***
(0.23) & NA
(NA) & -1.23***
(0.14)\\
N respondents per node &  & 0.18***
(0.01) & 0.29***
(0.06) & NA
(NA) & 0.14***
(0.04)\\
\addlinespace
N generic connections &  & -0.01
(0.01) & -0.08***
(0.03) & NA
(NA) & -0.03
(0.02)\\
Edges & -5.11***
(0.04) & -5.49***
(0.1) & -6.68***
(0.28) & NA
(NA) & -7.61***
(0.26)\\
Fixed -Inf &  & -Inf***
(0) & -Inf***
(0) & -Inf***
(0) & -Inf***
(0)\\
BIC & 9978 & 8149 & 5817 & 7584 & 5648\\
\bottomrule
\end{tabular}}
\end{table}
\begin{table}

\caption{\label{tab:unnamed-chunk-43}ERGM comparison for valuation resource subsystem}
\centering
\resizebox{\linewidth}{!}{
\begin{tabular}[t]{llllll}
\toprule
Coefficient & Model1 & Model2 & Model3 & Model4 & Model5\\
\midrule
Anti-centralization &  & 3.62***
(0.41) & 3.81***
(0.42) & NA
(NA) & 4.36***
(0.47)\\
Triadic closure &  & 0.95***
(0.03) & 0.35***
(0.06) & NA
(NA) & 0.39***
(0.06)\\
Actor-type: Government &  &  &  & NA
(NA) & 1.39***
(0.17)\\
Actor-type: University \& extension &  &  &  & NA
(NA) & -0.4**
(0.19)\\
Actor-type: Organization &  &  &  & NA
(NA) & -0.55***
(0.15)\\
\addlinespace
Actor-type: Company &  &  &  & NA
(NA) & 0.5***
(0.12)\\
Regional homophily: Northeast &  &  & 2.5***
(0.34) & NA
(NA) & 2.38***
(0.37)\\
Regional homophily: South &  &  & 1.34***
(0.28) & NA
(NA) & 1.37***
(0.3)\\
Regional homophily: North Central &  &  & 0.9***
(0.24) & NA
(NA) & 1.31***
(0.23)\\
Regional homophily: West &  &  & 1.43***
(0.17) & NA
(NA) & 1.39***
(0.16)\\
\addlinespace
National homophily: US National &  &  & -0.5
(0.35) & NA
(NA) & -0.68**
(0.3)\\
Scale: National \& International (vs. Regional) &  &  & 1.25***
(0.12) & NA
(NA) & 1.18***
(0.12)\\
Multi-functional tie &  &  & 5.23***
(0.12) &  & 5.48***
(0.16)\\
Survey non-respondent &  & -0.98***
(0.1) & -0.49***
(0.16) & NA
(NA) & -0.83***
(0.16)\\
N respondents per node &  & 0.07***
(0.02) & 0.02
(0.06) & NA
(NA) & 0.15**
(0.07)\\
\addlinespace
N generic connections &  & 0.02***
(0.01) & 0.04***
(0.01) & NA
(NA) & 0.02
(0.01)\\
Edges & -4.9***
(0.04) & -4.95***
(0.09) & -6.53***
(0.22) & NA
(NA) & -7.05***
(0.29)\\
Fixed -Inf &  & -Inf***
(0) & -Inf***
(0) & -Inf***
(0) & -Inf***
(0)\\
BIC & 7606 & 6651 & 4643 & 6384 & 4516\\
\bottomrule
\end{tabular}}
\end{table}
For the full and final model, with decay values of gwdegree alpha = 0.1 and gwesp alpha = 0.9, we can look at goodness-of-fit plots.
\begin{figure}

{\centering \includegraphics[width=1\linewidth]{../osisn_processes/figures/figurec2} 

}

\caption{Knowledge resource subsystem ERGM goodness of fit plots}\label{fig:unnamed-chunk-44}
\end{figure}
\begin{figure}

{\centering \includegraphics[width=1\linewidth]{../osisn_processes/figures/figurec3} 

}

\caption{Valuation resource subsystem ERGM goodness of fit plots}\label{fig:unnamed-chunk-45}
\end{figure}
\hypertarget{full-model-results-1}{%
\section{4. Full model results}\label{full-model-results-1}}
\begin{table}

\caption{\label{tab:unnamed-chunk-46}Full model results}
\centering
\resizebox{\linewidth}{!}{
\begin{tabular}[t]{lrrrr}
\toprule
\multicolumn{1}{c}{ } & \multicolumn{2}{c}{Knowledge subsystem} & \multicolumn{2}{c}{Valuation subsystem} \\
\cmidrule(l{3pt}r{3pt}){2-3} \cmidrule(l{3pt}r{3pt}){4-5}
Term & Estimate & SE & Estimate & SE\\
\midrule
Anti-centralization & 5.309 & 0.521 & 4.363 & 0.466\\
Triadic closure & 0.769 & 0.059 & 0.389 & 0.056\\
Actor-type: Government & -0.140 & 0.279 & 1.393 & 0.171\\
Actor-type: University \& extension & 1.131 & 0.126 & -0.395 & 0.194\\
Actor-type: Organization & 0.932 & 0.127 & -0.553 & 0.152\\
\addlinespace
Actor-type: Company & -0.129 & 0.124 & 0.497 & 0.124\\
Regional homophily: Northeast & 2.596 & 0.230 & 2.381 & 0.374\\
Regional homophily: South & 1.958 & 0.163 & 1.369 & 0.299\\
Regional homophily: North Central & 1.846 & 0.143 & 1.309 & 0.233\\
Regional homophily: West & 1.436 & 0.152 & 1.393 & 0.165\\
\addlinespace
National homophily: US National & -0.898 & 0.484 & -0.685 & 0.297\\
Scale: National \& International (vs. Regional) & 0.908 & 0.116 & 1.179 & 0.117\\
Multi-functional tie & 5.767 & 0.168 & 5.483 & 0.161\\
Survey non-respondent & -1.231 & 0.141 & -0.829 & 0.156\\
N respondents per node & 0.140 & 0.038 & 0.148 & 0.070\\
\addlinespace
N generic connections & -0.026 & 0.016 & 0.020 & 0.015\\
Edges & -7.607 & 0.260 & -7.050 & 0.294\\
Fixed -Inf & -Inf & 0.000 & -Inf & 0.000\\
\bottomrule
\end{tabular}}
\end{table}
\hypertarget{colophon}{%
\chapter*{Colophon}\label{colophon}}
\addcontentsline{toc}{chapter}{Colophon}

This document is set in \href{https://github.com/georgd/EB-Garamond}{EB Garamond}, \href{https://github.com/adobe-fonts/source-code-pro/}{Source Code Pro} and \href{http://www.latofonts.com/lato-free-fonts/}{Lato}. The body text is set at 11pt with \(\familydefault\).

It was written in R Markdown and \(\LaTeX\), and rendered into PDF using \href{https://github.com/ryanpeek/aggiedown}{aggiedown} and \href{https://github.com/rstudio/bookdown}{bookdown}.

This document was typeset using the XeTeX typesetting system, and the University of California Thesis class. Under the hood, the elements of the document formatting source code have been taken from the \href{https://github.com/stevenpollack/ucbthesis}{Latex, Knitr, and RMarkdown templates for UC Berkeley's graduate thesis}, and \href{https://github.com/suchow/Dissertate}{Dissertate: a LaTeX dissertation template to support the production and typesetting of a PhD dissertation at Harvard, Princeton, and NYU}

The source files for this thesis have been compiled at \url{https://github.com/liza-wood/aggiedown_dissertation}.

\backmatter

\hypertarget{references}{%
\chapter*{References}\label{references}}
\addcontentsline{toc}{chapter}{References}

\markboth{References}{References}

\noindent

\end{ucmainmatter}
\end{document}

%---Set Headers and Footers ------------------------------------------------------
\pagestyle{fancy}
\renewcommand{\chaptermark}[1]{\markboth{{\sf #1 \hspace*{\fill} Chapter~\thechapter}}{} }
\renewcommand{\sectionmark}[1]{\markright{ {\sf Section~\thesection \hspace*{\fill} #1 }}}
\fancyhf{}

\makeatletter \if@twoside \fancyhead[LO]{\small \rightmark} \fancyhead[RE]{\small\leftmark} \else \fancyhead[LO]{\small\leftmark}
\fancyhead[RE]{\small\rightmark} \fi

\def\cleardoublepage{\clearpage\if@openright \ifodd\c@page\else
  \hbox{}
  \vspace*{\fill}
  \begin{center}
    This page intentionally left blank
  \end{center}
  \vspace{\fill}
  \thispagestyle{plain}
  \newpage
  \fi \fi}
  
\makeatother
\fancyfoot[c]{\textrm{\textup{\thepage}}} % page number
\fancyfoot[C]{\thepage}
\renewcommand{\headrulewidth}{0.4pt}

\fancypagestyle{plain} { \fancyhf{} \fancyfoot[C]{\thepage}
\renewcommand{\headrulewidth}{0pt}
\renewcommand{\footrulewidth}{0pt}}
